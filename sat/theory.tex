\section{Theoretical Background}

%\subsection{Classical Planning -- Recap}


%\begin{frame}{Classical Planning}
%
%    $\mathcal P = (V,A,s_I,g)$
%    \begin{itemize}
%	 \item $V$ a set of (binary) state variables.
%	 \item $A \subseteq 2^V \times 2^V \times 2^V$ a set of actions.
%     \item $s_I \subseteq V$ the initial state.
%     \item $g \subseteq V$ the goal.
%    \end{itemize}\ \\[1em]
%	
%    \pause 
%	A solution $\pi = (a_1, \dots , a_n)$ has
%    \begin{itemize}
%		\item to be executable in $s_I$ and
%		\item to result in a state $s' \supseteq g$.
%    \end{itemize}\ \\[1em]
%
% \begin{tikzpicture}[scale=1.0]
%    \tikzstyle{A}=[draw,circle,minimum size=4pt,inner sep=0pt]
%    \tikzstyle{C}=[A,fill=black]
%    
%    \newcommand*\GFA{}
%    \newcommand*\GFB{}
%    \newcommand*\GFC{}
%    \newcommand*\GFD{}
%
%    %% keep the thing from moving
%    % \node[] at (0,5) {};
%    % \node[] at (8,-2) {};
%    
%    \draw [fill] (-4/6,-1) rectangle (-4/6-0.03,-2);
%    \node [] at (-4/6,-.8) {$\scriptstyle s_I$};
%    \node [] (EP0) at (-4/6-0.1,-1.5) {};
%  
%    \draw [fill] (7*4/6,-1) rectangle (7*4/6-0.03,-2);
%    \node [] at (7*4/6,-.8) {$\scriptstyle s' \supseteq g$};
%    \node [] (EP8) at (7*4/6+0.09,-1.5) {};
%    
%    \foreach \x in {1,...,7}
%      \pgfmathtruncatemacro{\xminusone}{\x - 1}
%      \node [C] (EP\x) at (4/6*\xminusone,-1.5) {};
%    \foreach \x in {0,...,7}	
%      \pgfmathtruncatemacro{\xplusone}{\x + 1}
%      \draw [->] (EP\x) -- (EP\xplusone);
%  \end{tikzpicture}
%\end{frame}






%\subsection{Complexity}


\begin{frame}{Computational Complexity}
	\begin{definition}[\textsc{PlanEx}]
		Given a planning problem $\mathcal P$.\\
		Is there a solution $\pi$ of $\mathcal P$.
	\end{definition}
	\begin{theorem}<+(1)->[Bylander'94]
		\textsc{PlanEx} is $\mathbb{PSPACE}$-complete.
	\end{theorem}

	\begin{theorem}<+(1)->[Bylander'94]
		\textsc{PlanEx} with bounded plan length $k$ is $\mathbb{PSPACE}$-complete.
	\end{theorem}

	\centering\ \\[.25em]
	\visible<+(1)->{$\mathbb{PSPACE}$ with $\mathbb{NP}$ calculus?}
\end{frame}






\subsection{Bridging the Gap between $\mathbb{NP}$ and $\mathbb{PSPACE}$}


\begin{frame}{Transformation Idea}
	\begin{itemize}[<+->]
		\item Bounded plan length assumes binary encoding of $k$.
		\item What if we assume $k$ in \emph{unary} encoding?
		\item \textsc{PlanEx} ``becomes'' $\mathbb{NP}$-``complete''.
		\item For full \textsc{PlanEx}: how to choose the plan length?
		\begin{itemize}
			\item Theoretical limit: $2^{|V|}$.
			\item Practical limit: usually smaller (sometimes polynomially bounded).
			\item Work by Abdulaziz [AAAI'21]
		\end{itemize}
		\item{Start with a small $k$ and increase until a solution is found.}
	\end{itemize}
\end{frame}






\begin{frame}[label=BoundIteration]{Bound Iteration}
\centering
\scalebox{0.5}{\begin{tikzpicture}
    \node (X) at (-3,1.5) {\foldedpaper{1}};
	\node at (-3,0.4) {Planning Problem};
    
	\draw[thick,line width=0.3mm,->] (-2,1.5) -- (-0.25,1.5);
	\node at (16.5,4.5) {};
	\node at (16.5,-4.1) {};

    \node at (1,1) {\gear{10}{1.8}{2.4}{10}{2}{0.8}{0.2}};
    \node at (1.9,1.2) {\gear{10}{1.8}{2.4}{10}{2}{0.8}{0.2}};
    \draw (3,3) rectangle (0,0);
    \node at (1.5,2.5) {Transformer};
    \node<1-3> at (1.5,2.0) {$k=1$};
    \node<4> at (1.5,2.0) {$k=2$};
    \node<5> at (1.5,2.0) {$k=3$};
    \node<6> at (1.5,2.0) {$k=\dots$};
    \node<7> at (1.5,2.0) {$k=2^{|V|}$};
    
    \draw[thick,line width=0.3mm,->] (3.25,1.5) -- (5,1.5);

	\node (X) at (6,1.5) {\foldedpaper{1}};
	\node at (6,0.4) {SAT problem};
    
	\draw[thick,line width=0.3mm,->] (7,1.5) -- (8.75,1.5);

    \node at (10,1) {\color{red}\gear{10}{1.8}{2.4}{10}{2}{0.8}{0.2}};
    \node at (10.9,1.2) {\color{red}\gear{10}{1.8}{2.4}{10}{2}{0.8}{0.2}};
    \draw (12,3) rectangle (9,0);
    \node at (10.5,2.5) {SAT Solver};
	
	\draw<2->[thick,line width=0.3mm,->] (12.25,1.75) -- (14, 3.5);
	\draw<2->[thick,line width=0.3mm,->] (12.25,1.25) -- (14,-0.5);
	\node<2-> (X) at (15,3.5) {\foldedpaper{1}};
	\node<2-> (X) at (15,2.4) {Solution};
	\node<2-> (X) at (15,-0.5) {\scalebox{4}{$\emptyset$}};
	\node<2-> (X) at (15,-1.4) {Unsolvable};

	\draw<3->[thick,line width=0.3mm,->] (15,-1.8) to [bend left] (1.5,-0.5);
\end{tikzpicture}}
\end{frame}


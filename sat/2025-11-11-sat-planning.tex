%!TeX program = xelatex

\documentclass[10pt,american,aspectratio=169,xcolor=table]{beamer}

% Languages and fonts
\usepackage[utf8]{inputenc}
\usepackage[T2A,T1]{fontenc}

\usepackage{amsmath,amssymb}
\usepackage{mathtools}
\usepackage[all]{foreign}

\usepackage[
  style=numeric, maxcitenames=99, maxbibnames=99, 
  bibencoding=utf8, language=auto, autolang=other,
  sorting=none
]{biblatex}

\usetheme{formally}

\addbibresource{biblio.bib}

\title{%
  An introduction to\\
  Planning as Satisfiability
}

\IfFileExists{config.cfg}{\input{config.cfg}}

\author{%
  \texorpdfstring{%
    \begingroup
      \renewcommand\author[1]{{\small ##1}}%
      \newcommand\speaker[1]{{\small\alert{##1}}}%
      \newcommand\affil[1]{{\scriptsize ##1}}%
      \begin{tabular}{@{} c c c c @{}}
        \speaker{Gregor Behnke} & 
        \author{Matteo Cardellini} & 
        \author{Nicola Gigante} & 
        \author{Andrea Micheli}\\
        \affil{University of Amsterdam} & 
        \affil{University of Genoa }& 
        \affil{Free University of Bozen-Bolzano }& 
        \affil{Fondazione Bruno Kessler}\\
        \affil{The Netherlands }& 
        \affil{Italy }& 
        \affil{Italy }& 
        \affil{Trento, Italy}
      \end{tabular}% 
    \endgroup
  }{%
    Nicola Gigante
  }%
}

\date{%
  ICAPS 2025\\
  November 10\\
  Melbourne, Australia
}






\usepackage[headheight=12pt,footheight=12pt]{beamerthemeboxes}

\usepackage[english]{babel}
\usepackage[utf8]{inputenc}
\usepackage{amsmath,amssymb,amsfonts}
\usepackage{ulem}
\usepackage{times}
\usepackage{paralist}
\usepackage{graphicx}
\usepackage{fancyvrb}
\usepackage{array}
\usepackage{amsmath}
\usepackage{mathtools}
\usepackage{colortbl}
\usepackage{tikz}                        % (standard)
\usepackage{ifthen}
\usepackage{multicol}
\usepackage{vwcol}
\usepackage{tabularx}
\usepackage{subcaption}
\usepackage{booktabs}
\usepackage{arydshln}
\usepackage{transparent}
\usepackage{soul}
\usepackage{contour}
\usepackage{scalefnt}
\usepackage{mathtools}
\usepackage{booktabs}
\usepackage{adjustbox}
\usepackage[makeroom,thicklines]{cancel}
%\usepackage{fontawesome5}
\usepackage[absolute]{textpos}
 \setlength{\TPHorizModule}{1mm}%
 \setlength{\TPVertModule}{1mm}%

\usepackage{amsmath,amsthm,amssymb,stmaryrd}              % (standard)
\usepackage{multicol,multirow}           % (standard)
\usepackage{xparse}                      % allows to define own commands with multiple optional arguments
\usepackage{fontawesome}

\usepackage{etoolbox}
\usetikzlibrary{decorations.pathreplacing,snakes}

%%%%%%%%%%%%%% Pascals Farben
\newcommand{\ColorTop}    {uc_prime}
\newcommand{\ColorBotton} {\ColorTop}
\newcommand{\BeamerColor} {\ColorBotton}

\DeclareMathOperator*{\AMO}{\mathbb{M}}

\newcommand{\oftotalframenumber}{/\ref{last:slide}}

\usetikzlibrary{arrows,chains,decorations.pathmorphing,decorations.shapes,shapes.geometric,positioning,calc,shapes,scopes,backgrounds,fit,matrix,trees}
\usetikzlibrary{decorations.pathreplacing}
\tikzset{ onslide/.code args={<#1>#2}{\only<#1>{\pgfkeysalso{#2}}} }
% \definecolor{uc_prime}{RGB}{0,153,0}  % Standard-Farbe von Tasks
% \definecolor{uc_second}{RGB}{1,1,1}   % insertion arrows
% \definecolor{uc_accent}{RGB}{20,80,20}   % Lösungen

\newcommand{\LabelForInit} {init}
\newcommand{\LabelForGoal} {goals}

% \definecolor{uc_prime}{RGB}{125,154,170}  % modifiziert
% \definecolor{uc_second}{RGB}{163,38,56}   % insertion arrows
% \definecolor{uc_accent}{RGB}{169,162,141} % Lösungen

\pgfarrowsdeclare{biggertip}{biggertip}{  
  \setlength{\arrowsize}{0.03em}  
  \addtolength{\arrowsize}{.5\pgflinewidth}  
  \pgfarrowsrightextend{0}  
  \pgfarrowsleftextend{-5\arrowsize}  
}{  
  \setlength{\arrowsize}{0.03em}  
  \addtolength{\arrowsize}{.5\pgflinewidth}  
  \pgfpathmoveto{\pgfpoint{-5\arrowsize}{4\arrowsize}}  
  \pgfpathlineto{\pgfpointorigin}  
  \pgfpathlineto{\pgfpoint{-5\arrowsize}{-4\arrowsize}}  
  \pgfusepathqstroke  
}

\tikzset{
  math/.style={execute at begin node=$,execute at end node=$},
  tn/.style={draw,rounded corners=0.8em,minimum width=8.5em,fill=white},      %task network
  solution/.style={fill=uc_accent!30},
  modified/.style={fill=uc_prime!30},
  cut/.style={draw=black,text=black,fill=uc_prime!30},
  uncut/.style={draw=black,text=black,fill=white},
  task/.style={thick,math,minimum width=2em,draw,fill=white,text depth=0em,text height=0.7em},
  primitive/.style={rounded corners=0.2em},
  abstract/.style={rounded corners=0em},
  abstractm/.style={abstract,modified},
  primm/.style={primitive,modified},
  tprim/.style={task,primitive},                                        % primitive task
  tabstr/.style={task,abstract},                                        % abstract task
  insertion/.style={-latex,dashed,draw=uc_second,very thick},           % insertion arrows
  decomposition/.style={-biggertip,very thick,draw},
  oc/.style={-latex,thick},                                             % ordering constraints
  method/.style={rounded corners,draw=\ColorTop,ultra thick,rectangle},
  onslide/.code args={<#1>#2}{\only<#1>{\pgfkeysalso{#2}}}
}

\def\emptyDefault#1#2{\ifx&#1&%
#2\else#1\fi}

\newcommand{\TaskSeq}[1]{
   \path[start chain=1,every on chain/.style={join=by oc},node distance=0.9em]
     \foreach \x/\attr in {#1} {
       node[task,
          rectangle,
          on chain=1,
          primitive,
          \attr] {\x}};
}

\newcommand{\MInsert}[1] {Insertion: $#1$}
\newcommand{\MDec}[1]    {Decomposition: $#1$}

\newcommand{\StackTN}[2] {
  \tikz[start chain=going below, node distance=0]{
    \node[on chain,draw=none] {\tikz{\TaskSeq{#1}}};
    \node[on chain,draw=none] {\tikz{\TaskSeq{#2}}};
  }
}

\newcommand{\GF}{}    %comes after a symbol from the first grammar
\newcommand{\GS}{'}    %comes after a symbol from the second grammar

\newcommand{\BlueBullets}{
  \renewcommand{\labelitemi}{\textcolor{uc_prime}{$\bullet$}}
  \renewcommand{\labelitemii}{\textcolor{uc_prime}{$\blacktriangleright$}}
}
\newcommand{\BrownBullets}{\renewcommand{\labelitemi}{\textcolor{uc_accent}{$\bullet$}}}


\tikzstyle{heading}=[rounded corners=0.3em,inner xsep=0.6em,right=10pt,fill=uc_prime,text=white, text depth=0.1em,font=\bfseries]
\tikzstyle{box}=[
  rounded corners=1em,
  inner sep=1em,
  inner ysep=1.5em,
  fill=uc_prime!20,
  line width=5pt,
  draw=uc_prime!40,
  execute at begin node=\BlueBullets
]
\tikzstyle{comment}=[
  box,
  fill=uc_accent!20,
  draw=uc_accent!40,
  execute at begin node=\BrownBullets
]



\newcommand{\LegItem}[3]     {\node {$#2 \rightarrow$}; \pgfmatrixnextcell \node[tn] {\tikz{#3}};}
\newcommand{\LegListItem}[3] {\LegItem{#1}{#2}{\TaskSeq{#3}}}


%%%%%%%%%%%%%%%%%%%%%%%%%%%%%%%%%%%%%%%%%%%%%% symbols

\def\foldedpaper#1{
    \tikz[scale=#1,line width={#1*1pt}]{
        \def\a{1.41} % relative height
        \def\b{0.2}  % relative height/width of corner
        \def\c{0.1}  % relative margin width (on either side)
        \def\d{0.05} % vertical offset of lines
        \def\N{6}    % number of lines
        \draw         (0,0)
                --  ++(-1,0)
                --  ++(0,\a)
                --  ++(1-\b,0)
                --  ++(\b,-\b)
                -- cycle;
        \foreach \lnum in {1,...,\N}{
            \pgfmathsetmacro\yline{\a-\d-\lnum*\a/(\N+1)}
            \draw (-1+\c,\yline) -- (-\c,\yline);
        }
        \draw[fill=white] (0,\a-\b) -- ++(-\b,0) -- ++ (0,\b);
    }
}

% #1 number of teeths
% #2 radius intern
% #3 radius extern
% #4 angle from start to end of the first arc
% #5 angle to decale the second arc from the first
% #6 inner radius to cut off

\newcommand{\gear}[7]{
\tikz[scale=#7,line width={#7*1pt}]{%
	\fill[even odd rule]
  (0:#2)
  \foreach \i [evaluate=\i as \n using {\i-1)*360/#1}] in {1,...,#1}{%
    arc (\n:\n+#4:#2) {[rounded corners=1.5pt] -- (\n+#4+#5:#3)
    arc (\n+#4+#5:\n+360/#1-#5:#3)} --  (\n+360/#1:#2)
  }%
  (0,0) circle[radius=#6]
  }
}


\newcommand{\manfig}[1]{
\tikz[scale=#1,line width={#1*1pt}]{%
\node[circle,fill,minimum size=5mm] (head) {};
\node[rounded corners=2pt,minimum height=1.3cm,minimum width=0.4cm,fill,below = 1pt of head] (body) {};
\draw[line width=1mm,round cap-round cap] ([shift={(2pt,-1pt)}]body.north east) --++(-90:6mm);
\draw[line width=1mm,round cap-round cap] ([shift={(-2pt,-1pt)}]body.north west)--++(-90:6mm);
\draw[thick,white,-round cap] (body.south) --++(90:5.5mm);
}
}




% \include{AssemblyTaskDomainDefinition}





\newcommand\todo[1]{\textbf{\textcolor{red}{TODO: {#1}}}}

\newcommand{\Hyb}              {\textsc{TiHtn}}
\newcommand{\HTN}              {\textsc{Htn}}
\newcommand{\POCL}             {\textsc{Pocl}}

% HTN subclasses
\newcommand{\htn}              {HTN}
\newcommand{\tihtn}            {TIHTN}%{\ensuremath{\mathit{HTN_{TI}}}}
\newcommand{\htnnoloop}        {\ensuremath{\mathit{HTN_{acyc}}}}
\newcommand{\htnordered}       {\ensuremath{\mathit{HTN_{ord}}}}
\newcommand{\htnall}           {\ensuremath{\mathit{\htn}}}
\newcommand{\noeff}            {\ensuremath{0\,\mathit{eff}}}
\newcommand{\nopre}            {\ensuremath{0\,\mathit{pre}}}
\newcommand{\htnnope}          {\ensuremath{\htn_{\noeff}^{\nopre}}} % HTN without preconditions and effects
\newcommand{\tihtnnope}        {\ensuremath{\tihtn_{\noeff}^{\nopre}}}
\newcommand{\minl}             {\ensuremath{ m }}
\newcommand{\ExePlan}          {\ensuremath{\mathit{EXE}}}

\newcommand{\Lhyb}             {\ensuremath{L_\Hyb}}
\newcommand{\Lhtn}             {\ensuremath{L_\HTN}}
\newcommand{\Lexe}             {\ensuremath{L_{prim}}}
\newcommand{\Ls}               {\ensuremath{L_{\textsc{S}}}}


\newcommand{\reg}{\ensuremath{\mathit{REG}}} % regular language
\newcommand{\cfl}{\ensuremath{\mathit{CF}}} % context-free language
\newcommand{\csl}{\ensuremath{\mathit{CS}}} % context-sensitive language
\newcommand{\idlp}{\textsc{Id/Lp}} 

\newcommand{\lba}{linear bounded automaton} 
\newcommand{\lbaabr}{\textsc{Lba}} 
\newcommand{\parser}{acceptor} 

% basic definitions
%
\newcommand{\Prob}             {\ensuremath{\mathcal{P}}}
\newcommand{\methods}          {\ensuremath{M}}
\newcommand{\allTNsX}          {\ensuremath{\mathit{TN}_{\hspace{-0.1em}N}}}
\newcommand{\allTNsP}          {\ensuremath{\mathit{TN}_{\hspace{-0.1em}O}}}
\newcommand{\allTNsC}          {\ensuremath{\mathit{TN}_{\hspace{-0.1em}C \cup O}}}
\newcommand{\Prec}[1]          {\text{prec}_{#1}}   % operator precondition
\newcommand{\Add}[1]           {\text{add}_{#1}}    % operator add list
\newcommand{\Del}[1]           {\text{del}_{#1}}    % operator delete list
\newcommand{\tn}               {\mathrm{tn}}
\newcommand{\ti}               {c_I}
\newcommand{\OrdUn}            {\ensuremath{\mathord{\prec}}}
\newcommand{\sol}[1]           {\ensuremath{Sol_{#1}}}
\newcommand{\yield}            {\mathrm{yield}}
\newcommand{\Goal}             {\mathrm{goal}}
\newcommand{\Edges}            {E}
\newcommand{\Children}[2]      {\ensuremath{\mathit{ch}(#2,#1)}}   %1==task, 2==tree
\newcommand{\Ind}[2]           {\ensuremath{\mathit{ind}(#2,#1)}}  %1==task, 2==tree

\newcommand{\phat}             {\ensuremath{\mathcal P_{\text{no-\textsc{h}}}}}
\newcommand{\pzero}            {\ensuremath{\mathcal P_{\text{no-\textsc{pe}}}}}

% technical definitions
%
\newcommand{\Shortcite}[1]     {\citeauthor{#1}~\shortcite{#1}}
\newcommand{\with}             {\mid}
\newcommand{\restrict}[1]      {|_{#1}}
\newcommand{\card}[1]          {\left| #1 \right|}
\newcommand{\cardSigma}[1]     {\left| #1 \right|_\Sigma}
\newcommand{\defeq}            {\mathbin{:=}}
\newcommand{\stsubst}[3]       {#1 [ #2 \! \leftarrow \! #3]}


% complexity classes
%
\newcommand{\ClassP}               {\ensuremath{\mathbb{P}}}
\newcommand{\ClassNP}              {\ensuremath{\mathbb{NP}}}
\newcommand{\ClassNPComplete}      {\ensuremath{\mathbb{NP-complete}}}
\newcommand{\ClassNPHard}          {\ensuremath{\mathbb{NP-hard}}}
\newcommand{\ClassPSPACE}          {\ensuremath{\mathbb{\textsc{Pspace}}}}
\newcommand{\ClassPSPACEComplete}  {\ensuremath{\mathbb{\textsc{Pspace}-complete}}}
\newcommand{\ClassEXPSPACE}        {\ensuremath{\mathbb{\textsc{Expspace}}}}
\newcommand{\ClassNEXPTIME}        {\ensuremath{\mathbb{NEXP}}-\textsc{time}}


\newcommand{\ONF} {NF$_{\neq1}$}
\newcommand{\TNF} {NF$_{\geq2}$}

\newcommand{\scalefactor} {0.9125}
\newcommand{\hierarch} {\HTN} % seems to be redundant

\newcommand{\figabr}[1]{Fig.\,#1}
\newcommand{\algabr}[1]{Alg.\,#1}
\newcommand{\thmabr}[1]{Thm.\,#1}
\newcommand{\lemabr}[1]{Lem.\,#1}
\newcommand{\propabr}[1]{Prop.\,#1}
\newcommand{\defabr}[1]{Def.\,#1}
\newcommand{\corabr}[1]{Cor.\,#1}
\newcommand{\tababr}[1]{Tab.\,#1}
\newcommand{\eqabr}[1]{Eq.\,#1}

% for parser
\newcommand{\tape}[1]{\textsc{Tape-#1}}
\newcommand{\head}[1]{\textsc{Head-#1}}
\newcommand{\myfail}[0]{\normalfont  $\mathit{failure}$}
\newcommand{\mysuccess}[0]{\normalfont  $\mathit{success}$}
\newcommand{\tapeend}[0]{\textsc{end}}
\newcommand{\ordok}{\textsc{ok}}
\newcommand{\treesym}{\ensuremath{o}}
\newcommand{\pzwei}{$m_{\treesym}$}
\newcommand{\peins}{$m_{p}$}

\newcommand{\slideTi}[0]{2,3}
\newcommand{\slideCfWithoutOrderedHTN}[0]{1,2,3}
\newcommand{\slideNoLoop}[0]{3}
\newcommand{\slideCf}[0]{0}
\newcommand{\slideNoop}[0] {0}
\newcommand{\slideAll}[0] {0}

\newcommand{\erol}{Erol et al., 'Complexity results for HTN planning', Annals of Mathematics and Artificial Intelligence, 1996}
\newcommand{\geier}{Geier and Bercher, 'On the Decidability of \htn{} Planning with Task Insertion', Int. Joint Conf. on AI, 2011}

\newcommand{\TN}               {\mathrm{tn}}                  % TN == Task Network
\newcommand{\PO}               {\ensuremath{\mathord{\prec}}} % PO == Partial Order
\newcommand{\NEWPrec}             {\text{prec}}   % operator precondition
\newcommand{\NEWAdd}              {\text{add}}    % operator add list
\newcommand{\NEWDel}              {\text{del}}    % operator delete list
\newcommand{\AllTNs}           {\ensuremath{\mathit{TN}}}


%%%%%%%%%%%%%%%% figure macro
\newcommand*\ShDec{}
\newcommand*\ShPl{}
\newcommand*\ShN{}
\newcommand*\ShSeq{}

\newcommand{\setPlan}[4]{
		\renewcommand*\ShDec{#1}
  \renewcommand*\ShPl{#2}
  \renewcommand*\ShN{#3}
  \renewcommand*\ShSeq{#4}
}

\newcommand*\VCShow{}
\newcommand*\VCPhaseA{}
\newcommand*\VCPhaseB{}
\newcommand*\VCPhaseC{}
\newcommand*\VCPhaseD{}

%\definecolor{shadegrey}{RGB}{80,20,140}
\definecolor{shadegrey}{RGB}{140,140,140}



%\makeatletter
%\def\@cite#1#2{\footnotetext{[\textbf{#1\if@tempswa , #2\fi}]}}





%\NewCommandCopy{\oldPi}{\Pi}
%\renewcommand{\Pi}{\mathrm{\oldPi}}
%\NewCommandCopy{\oldDelta}{\Delta}
%\renewcommand{\Delta}{\mathrm{\oldDelta}}
%
%\NewCommandCopy{\oldfrown}{\frown}
%\renewcommand{\frown}{\vphantom{)}^\oldfrown}
%\newcommand{\bigfrown}{\mathop{\raisebox{1mm}{\text{\Large$\oldfrown$}}}}
%\newcommand{\equivalent}{\Longleftrightarrow}
%\newcommand{\define}{\underline}
%\newcommand\Tau{T}

%\newcommand{\set}[1]{\{\,#1\,\}}
%\DeclareRobustCommand{\sledom}{\mathrel{\text{\reflectbox{$\models$}}}}

\newcommand{\add}{\operatorname{add}}
\newcommand{\del}{\operatorname{del}}
\newcommand{\pre}{\operatorname{pre}}
\newcommand{\var}{\operatorname{var}}
\newcommand{\act}{\operatorname{act}}
\newcommand{\app}{\operatorname{app}}
\newcommand{\init}{\operatorname{init}}
\newcommand{\goal}{\operatorname{goal}}
\newcommand{\dom}{\operatorname{dom}}
\newcommand{\rge}{\operatorname{rge}}
\newcommand{\rel}{\operatorname{rel}}

\newcommand{\addof}[1]{{#1}_{\boldsymbol{\add}}}
\newcommand{\delof}[1]{{#1}_{\boldsymbol{\del}}}
\newcommand{\preof}[1]{{#1}_{\boldsymbol{\pre}}}
\newcommand{\piof}[1]{{#1}_{\boldsymbol{\pi}}}
\newcommand{\nuof}[1]{{#1}_{\boldsymbol{\nu}}}

\DeclareFontFamily{U}{mathb}{\hyphenchar\font45}
\DeclareFontShape{U}{mathb}{m}{n}{
<-6> mathb5 <6-7> mathb6 <7-8> mathb7
<8-9> mathb8 <9-10> mathb9
<10-12> mathb10 <12-> mathb12
}{}
\DeclareSymbolFont{mathb}{U}{mathb}{m}{n}
\DeclareMathSymbol{\llcurly}{\mathrel}{mathb}{"CE}
\DeclareMathSymbol{\ggcurly}{\mathrel}{mathb}{"CF}

\newcommand{\removesomevspace}{\vspace*{-2pt}}




%%%%%%%%%%%%%%%%%%%%%%%%%%%

\AtBeginSection[]{
  \begin{frame}
  \vfill
  \centering
  \begin{beamercolorbox}[sep=8pt,center,shadow=true,rounded=true]{title}
    \usebeamerfont{title}\insertsectionhead\par%
  \end{beamercolorbox}
  \vfill
  \end{frame}
}


%\newcommand{\unilogo}{\vspace*{-1.5cm}\includegraphics[height=2.05cm]{illc_no_text_logo.jpg}}
%%%%%%%%%%%%%%%%%%%%%%%%%%%%%%%%%%%%%%%%%%%%%%%%%%%%%%%%%%%%%%%%%



%%%%%%%%%%%%%%%%%%%%%%%%%%%
% Anfang der Folien
\begin{document}
%\preto\frame{\setcounter{footnote}{0}}


\frame[plain]{\titlepage}

% Das Inhaltsverzeichnis
% \hspace*{-0.7cm}
% \begin{frame}
%   \frametitle{Inhaltsverzeichnis}
%   \tableofcontents
% \end{frame}
%\newcommand{\coldot}[1]{\tikz\draw[#1,fill=#1] (0,0) circle (.5ex);}


%\newcommand{\HL}{white}


\begin{frame}{Introduction}
\pause
\begin{center}
        Why are we even trying to solve planning problems \textbf{ourselves}?\\[1cm]
                \pause
        Why can't we have other people solve them \textbf{for us}?
\end{center}
\end{frame}


\begin{frame}{Agenda}
\tableofcontents
\end{frame}


\newcommand{\Wider}[2] {\begin{minipage}{#1\textwidth}#2\end{minipage}}


% Pascal's definitions
\tikzset{%
  action/.style = {rectangle,rounded corners=0.2em,fill=black,minimum width=3mm},
  P/.style      = {action},                                          % P = primitive
  C/.style      = {action,draw=main_color,fill=white},                    % C = complex  
  A/.style      = {C},                                               % A = abstract (=complex)
  S/.style      = {circle,fill=black,minimum size=4pt,inner sep=0pt} % S = state
}



\section{SAT Modelling}


\subsection{Problem Solving}


\begin{frame}{Idea: Problem Transformation}
\centering

\scalebox{0.6}{\begin{tikzpicture}
    \node (X) at (-3,1.5) {\foldedpaper{1}};
	\node at (-3,0.4) {Planning Problem};
    
	\draw[thick,line width=0.3mm,->] (-2,1.5) -- (-0.25,1.5);
	\node at (12,3) {};

    \node at (1,1) {\gear{10}{1.8}{2.4}{10}{2}{0.8}{0.2}};
    \node at (1.9,1.2) {\gear{10}{1.8}{2.4}{10}{2}{0.8}{0.2}};
    \draw (3,3) rectangle (0,0);
    \node at (1.5,2.5) {\only<2->{Transformer}\only<1>{Planner}};
    
    \draw[thick,line width=0.3mm,->] (3.25,1.5) -- (5,1.5);

	\node (X) at (6,1.5) {\foldedpaper{1}};
	\node at (6,0.4) {\only<4>{SAT problem}\only<2-3>{XYZ Problem}\only<1>{Plan}};
    
	\draw<3->[thick,line width=0.3mm,->] (7,1.5) -- (8.75,1.5);

    \node<3-> at (10,1) {\color{red}\gear{10}{1.8}{2.4}{10}{2}{0.8}{0.2}};
    \node<3-> at (10.9,1.2) {\color{red}\gear{10}{1.8}{2.4}{10}{2}{0.8}{0.2}};
    \draw<3-> (12,3) rectangle (9,0);
    \node<3-> at (10.5,2.5) {\only<3>{XYZ}\only<4->{SAT} Solver};
\end{tikzpicture}}
\end{frame}





\subsection{SAT}


\begin{frame}[fragile]{SAT}
 	\begin{definition}[SAT]
		Given a propositional formula $\mathcal F$, decide whether $\mathcal F$ has a satisfying valuation.
	\end{definition}
 	\begin{definition}<2->[CNF-SAT]
		Given a propositional formula $\mathcal F$ in conjunctive normal form, decide whether $\mathcal F$ has a satisfying valuation.
	\end{definition}
	\visible<3->{A valuation is an assignment of decision variables to $\{\top,\bot\}$.}\\[.25em]
	\visible<4->{CNF:}
	\[\visible<4->{\mathcal F = \bigwedge_{C \in \mathfrak C} \bigvee_{\ell \in C} \ell}\]
	\visible<4->{\vspace*{-1em}\begin{center}($\mathfrak{C}$ is the set of clauses; $C$ is a clause, a set of literals.)\end{center}}
	
\end{frame}






\subsection{SAT Solvers}


\begin{frame}[fragile]{SAT Solvers}
\begin{itemize}[<+->]
  \item SAT solvers are programs that determine whether a satisfying valuation exists and if so output it.
  \item A \textbf{lot} of research in recent years \\ (annual competitions since 2002).
  \item Usable OSes have \texttt{minisat} in their package manager.
  \item Standardised input format DIMACS:\\[0.3cm]
\begin{minipage}[t]{0.2\textwidth}\begin{verbatim}
p cnf 5 3
1 -5 4 0
-1 5 3 4 0
-3 -4 0
  \end{verbatim}\end{minipage} \quad \raisebox{-.75cm}{$\equiv$} \quad\ 
\begin{minipage}[t]{0.5\textwidth}%
CNF with 5 vars and 3 clauses:\\
$(v_1\vee\neg v_5\vee v_4)\ \wedge$\\
$(\neg v_1\vee v_5\vee v_3\vee v_4)\ \wedge$\\
$(\neg v_3\vee\neg v_4)$
\end{minipage} \quad
\end{itemize}
\end{frame}

\begin{frame}[fragile]{SAT Solvers -- IPASIR}
\begin{itemize}[<+->]
  \item Most SAT solvers support the IPASIR interface: \texttt{ipasir\_add}, \texttt{ipasir\_solve}, \texttt{ipasir\_val}
  \item Also allows for incremental solving via \texttt{ipasir\_assume} (if supported)
  \item Some SAT solvers have their own interface \texttt{kissat\_add}, \texttt{kissat\_solve}, \texttt{kissat\_val}\\[0.5cm]
  \item PySAT (includes some pre-defined encodings)
\end{itemize}
\end{frame}







\renewcommand*\VCShow{black}
\newcommand*\VCShoww{black}
\subsection{Modelling Example}




\begin{frame}{Colouring}
\begin{definition}
    Given a graph $G=(V,E)$ and a number $k$.\\
	Is there an assignment of $k$ colours to the vertices of $G$, such that all adjacent vertices have different colours?
\end{definition}
   
\begin{center}
     \only<1>{\renewcommand*\VCShow{black}}
     \only<1>{\renewcommand*\VCShoww{black}}
     \only<2->{\renewcommand*\VCShow{red}}
     \only<2->{\renewcommand*\VCShoww{blue}}
     \scalebox{0.75}{
 \begin{tikzpicture}[scale=1.0]
    \tikzstyle{N}=[draw,circle,fill=black,minimum size=4pt,inner sep=0pt]
    \draw (0,0) node (a) [N,label=left:$a$] {}
          to node[midway,auto] {$e_{ab}$} ++(1.25cm,1.0cm) node (b) [N,label=above:$b$,\VCShow] {}
          to node[midway,auto] {$e_{bc}$} ++(1.25cm,1.0cm) node (c) [N,label=above:$c$] {}
          to node[midway,auto] {$e_{cg}$} ++(1.25cm,-1.0cm) node (g) [N,label=right:$g$,\VCShow] {}
          to node[midway,auto] {$e_{gd}$} ++(-1.25cm,-1.0cm) node (d) [N,label=below:$d$,\VCShoww] {}
          to node[midway,right] {$e_{dc}$} (c)
          ;
   \draw (b) to node[midway,below] {$e_{bd}\hspace*{0.5cm}$} (d);
 \end{tikzpicture}
}

    \end{center}
\end{frame}






\begin{frame}{Colouring}
Variables for choosing the colour of each node
\[\mathtt{colour}_v^i \text{ where } v\in V \text{ and } i \in \{1,\dots,k\}\]
\pause
If a node has a colour, all adjacent nodes have a different colour
\[\mathtt{colour}_v^i \rightarrow \neg \mathtt{colour}_{w}^i \quad \quad \forall (v,w) \in E\]
\pause
\vspace{-0.5cm}
\[\neg \mathtt{colour}_v^i \vee \neg \mathtt{colour}_{w}^i \quad \quad \forall (v,w) \in E\]
\pause
Every node has a colour
\[\bigvee_{i=1}^k \mathtt{colour}_v^i \quad \quad \forall v \in V\]
\pause
Every node has at most one colour
\[\bigwedge_{i=1}^k \left[\mathtt{colour}_v^i \rightarrow \bigwedge_{j=1,i\neq j}^k \neg \mathtt{colour}_v^j \right] \quad \quad \forall v \in V\]
\end{frame}
	

\section{Theoretical Background}

%\subsection{Classical Planning -- Recap}


%\begin{frame}{Classical Planning}
%
%    $\mathcal P = (V,A,s_I,g)$
%    \begin{itemize}
%	 \item $V$ a set of (binary) state variables.
%	 \item $A \subseteq 2^V \times 2^V \times 2^V$ a set of actions.
%     \item $s_I \subseteq V$ the initial state.
%     \item $g \subseteq V$ the goal.
%    \end{itemize}\ \\[1em]
%	
%    \pause 
%	A solution $\pi = (a_1, \dots , a_n)$ has
%    \begin{itemize}
%		\item to be executable in $s_I$ and
%		\item to result in a state $s' \supseteq g$.
%    \end{itemize}\ \\[1em]
%
% \begin{tikzpicture}[scale=1.0]
%    \tikzstyle{A}=[draw,circle,minimum size=4pt,inner sep=0pt]
%    \tikzstyle{C}=[A,fill=black]
%    
%    \newcommand*\GFA{}
%    \newcommand*\GFB{}
%    \newcommand*\GFC{}
%    \newcommand*\GFD{}
%
%    %% keep the thing from moving
%    % \node[] at (0,5) {};
%    % \node[] at (8,-2) {};
%    
%    \draw [fill] (-4/6,-1) rectangle (-4/6-0.03,-2);
%    \node [] at (-4/6,-.8) {$\scriptstyle s_I$};
%    \node [] (EP0) at (-4/6-0.1,-1.5) {};
%  
%    \draw [fill] (7*4/6,-1) rectangle (7*4/6-0.03,-2);
%    \node [] at (7*4/6,-.8) {$\scriptstyle s' \supseteq g$};
%    \node [] (EP8) at (7*4/6+0.09,-1.5) {};
%    
%    \foreach \x in {1,...,7}
%      \pgfmathtruncatemacro{\xminusone}{\x - 1}
%      \node [C] (EP\x) at (4/6*\xminusone,-1.5) {};
%    \foreach \x in {0,...,7}	
%      \pgfmathtruncatemacro{\xplusone}{\x + 1}
%      \draw [->] (EP\x) -- (EP\xplusone);
%  \end{tikzpicture}
%\end{frame}






%\subsection{Complexity}


\begin{frame}{Computational Complexity}
	\begin{definition}[\textsc{PlanEx}]
		Given a planning problem $\mathcal P$.\\
		Is there a solution $\pi$ of $\mathcal P$.
	\end{definition}
	\begin{theorem}<+(1)->[Bylander'94]
		\textsc{PlanEx} is $\mathbb{PSPACE}$-complete.
	\end{theorem}

	\begin{theorem}<+(1)->[Bylander'94]
		\textsc{PlanEx} with bounded plan length $k$ is $\mathbb{PSPACE}$-complete.
	\end{theorem}

	\centering\ \\[.25em]
	\visible<+(1)->{$\mathbb{PSPACE}$ with $\mathbb{NP}$ calculus?}
\end{frame}






\subsection{Bridging the Gap between $\mathbb{NP}$ and $\mathbb{PSPACE}$}


\begin{frame}{Transformation Idea}
	\begin{itemize}[<+->]
		\item Bounded plan length assumes binary encoding of $k$.
		\item What if we assume $k$ in \emph{unary} encoding?
		\item \textsc{PlanEx} ``becomes'' $\mathbb{NP}$-``complete''.
		\item For full \textsc{PlanEx}: how to choose the plan length?
		\begin{itemize}
			\item Theoretical limit: $2^{|V|}$.
			\item Practical limit: usually smaller (sometimes polynomially bounded).
			\item Work by Abdulaziz [AAAI'21]
		\end{itemize}
		\item{Start with a small $k$ and increase until a solution is found.}
	\end{itemize}
\end{frame}






\begin{frame}[label=BoundIteration]{Bound Iteration}
\centering
\scalebox{0.5}{\begin{tikzpicture}
    \node (X) at (-3,1.5) {\foldedpaper{1}};
	\node at (-3,0.4) {Planning Problem};
    
	\draw[thick,line width=0.3mm,->] (-2,1.5) -- (-0.25,1.5);
	\node at (16.5,4.5) {};
	\node at (16.5,-4.1) {};

    \node at (1,1) {\gear{10}{1.8}{2.4}{10}{2}{0.8}{0.2}};
    \node at (1.9,1.2) {\gear{10}{1.8}{2.4}{10}{2}{0.8}{0.2}};
    \draw (3,3) rectangle (0,0);
    \node at (1.5,2.5) {Transformer};
    \node<1-3> at (1.5,2.0) {$k=1$};
    \node<4> at (1.5,2.0) {$k=2$};
    \node<5> at (1.5,2.0) {$k=3$};
    \node<6> at (1.5,2.0) {$k=\dots$};
    \node<7> at (1.5,2.0) {$k=2^{|V|}$};
    
    \draw[thick,line width=0.3mm,->] (3.25,1.5) -- (5,1.5);

	\node (X) at (6,1.5) {\foldedpaper{1}};
	\node at (6,0.4) {SAT problem};
    
	\draw[thick,line width=0.3mm,->] (7,1.5) -- (8.75,1.5);

    \node at (10,1) {\color{red}\gear{10}{1.8}{2.4}{10}{2}{0.8}{0.2}};
    \node at (10.9,1.2) {\color{red}\gear{10}{1.8}{2.4}{10}{2}{0.8}{0.2}};
    \draw (12,3) rectangle (9,0);
    \node at (10.5,2.5) {SAT Solver};
	
	\draw<2->[thick,line width=0.3mm,->] (12.25,1.75) -- (14, 3.5);
	\draw<2->[thick,line width=0.3mm,->] (12.25,1.25) -- (14,-0.5);
	\node<2-> (X) at (15,3.5) {\foldedpaper{1}};
	\node<2-> (X) at (15,2.4) {Solution};
	\node<2-> (X) at (15,-0.5) {\scalebox{4}{$\emptyset$}};
	\node<2-> (X) at (15,-1.4) {Unsolvable};

	\draw<3->[thick,line width=0.3mm,->] (15,-1.8) to [bend left] (1.5,-0.5);
\end{tikzpicture}}
\end{frame}


\section{Sequential Classical Planning in SAT}


\begin{frame}<1>{Classical Planning via SAT [Kautz\&Selman'92]}
  \vspace{-1cm}
	\begin{center}
      \begin{tikzpicture}
  \coordinate (nodeDifferenceH) at (1,0);
  \coordinate (nodeDifferenceV) at (0,.35);
  
  \node at (21,-12) {};
  \node at (10,-7.5) {};
  
  \node<1->[S] (S1) at (11.5,-9.5) {};
  \node<1->[S] (S2) at ($(S1)+1*(nodeDifferenceH)$) {};
  \node<1->[S] (S3) at ($(S1)+2*(nodeDifferenceH)$) {};
  \node<1->[S] (S4) at ($(S1)+3*(nodeDifferenceH)$) {};
  \node<1->[S] (S5) at ($(S1)+4*(nodeDifferenceH)$) {};
  \node<1->[S] (S6) at ($(S1)+5*(nodeDifferenceH)$) {};
  \node<1->[S] (S7) at ($(S1)+6*(nodeDifferenceH)$) {};
  \node<1->[S] (S8) at ($(S1)+7*(nodeDifferenceH)$) {};
  \draw<1->[fill] (10.5,-10.) rectangle (10.45,-9.);
  \node<1-> at (10.45,-8.8) {\tiny $I$};
  \draw<1->[fill] (19.5,-10.) rectangle (19.45,-9.);
  \node<1-> at (19.45,-8.8) {\tiny $G$};
  
  \draw<1->[->] (10.5,-9.5) -- (S1);
  \draw<1->[->] (S1) edge [] (S2);
  \draw<1->[->] (S2) edge [] (S3);
  \draw<1->[->] (S3) edge [] (S4);
  \draw<1->[->] (S4) edge [] (S5);
  \draw<1->[->] (S5) edge [] (S6);
  \draw<1->[->] (S6) edge [] (S7);
  \draw<1->[->] (S7) edge [] (S8);
  \draw<1->[->,shorten >=.5mm] (S8) edge [] (19.5,-9.5);
  
  % new: 1-3 was: 2-
  \node<1-3> at ($(11,-10.5)+1/2*(nodeDifferenceH)$) (label-S0) {\tiny $s_{1}$};
  \foreach \x in {1,...,8} {
    \edef\xPlusOne{\x}
    \pgfmathparse{int(\xPlusOne+1)}
    \edef\xPlusOne{\pgfmathresult}
    \node<1-3> at ($(label-S0)+\x*(nodeDifferenceH)$)   {\tiny $s_{\xPlusOne}$};
  }
  \node<1-3> at ($(label-S0)-(nodeDifferenceH)$) {\tiny $s_{I}=s_0$};
  
  % new: 1-2 was: 3-
  \node<1-2> at ($(11.5,-9.25)-1/2*(nodeDifferenceH)$) (label-A1) {\tiny $a_{1}$};
  \foreach \x in {1,...,8} {
    \edef\xPlusOne{\x}
    \pgfmathparse{int(\xPlusOne+1)}
    \edef\xPlusOne{\pgfmathresult}
    \node<1-2> at ($(label-A1)+\x*(nodeDifferenceH)$)   {\tiny $a_{\xPlusOne}$};
  }
  
  \foreach \x in {11.5,...,19.5} {
    \edef\upperIndex{\x}
    \pgfmathparse{int(\upperIndex-10.5)}
    \edef\upperIndex{\pgfmathresult}
    \node<3-> at ($(\x,-9.25)-1/2*(nodeDifferenceH)$) (label-A\upperIndex) {\tiny $a_{1}^{\upperIndex}$};
    \foreach \y in {1,...,4} {
      \edef\yPlusOne{\y}
      \pgfmathparse{int(\yPlusOne+1)}
      \edef\yPlusOne{\pgfmathresult}
      \node<3-> at ($(label-A\upperIndex)+\y*(nodeDifferenceV)$)    {\tiny $a_{\yPlusOne}^{\upperIndex}$};
    }
  }
  
  \foreach \x in {10,...,19} {
    \edef\upperIndex{\x}
    \pgfmathparse{int(\upperIndex-10)}
    \edef\upperIndex{\pgfmathresult}
    \node<4-> at ($(\x,-10.5)+1/2*(nodeDifferenceH)$) (label-V\upperIndex) {\tiny $v_{1}^{\upperIndex}$};
     \foreach \y in {1,...,3} {
      \edef\yPlusOne{\y}
      \pgfmathparse{int(\yPlusOne+1)}
      \edef\yPlusOne{\pgfmathresult}
      \node<4-> at ($(label-V\upperIndex)-\y*(nodeDifferenceV)$)    {\tiny $v_{\yPlusOne}^{\upperIndex}$};
    }
  }
\end{tikzpicture}

	\end{center}
	A plan is just a sequence of state transitions.
	\begin{itemize}
		\item ``Mechanics'' is identical in all timesteps.
		\item Just model one timestep and copy'n'paste.
		\item Edge constraints!
	\end{itemize}
\end{frame}




\begin{frame}{Decision Variables}
	\begin{center}
      \begin{tikzpicture}
  \coordinate (nodeDifferenceH) at (1,0);
  \coordinate (nodeDifferenceV) at (0,.35);
  
  \node at (21,-12) {};
  \node at (10,-7.5) {};
  
  \node<1->[S] (S1) at (11.5,-9.5) {};
  \node<1->[S] (S2) at ($(S1)+1*(nodeDifferenceH)$) {};
  \node<1->[S] (S3) at ($(S1)+2*(nodeDifferenceH)$) {};
  \node<1->[S] (S4) at ($(S1)+3*(nodeDifferenceH)$) {};
  \node<1->[S] (S5) at ($(S1)+4*(nodeDifferenceH)$) {};
  \node<1->[S] (S6) at ($(S1)+5*(nodeDifferenceH)$) {};
  \node<1->[S] (S7) at ($(S1)+6*(nodeDifferenceH)$) {};
  \node<1->[S] (S8) at ($(S1)+7*(nodeDifferenceH)$) {};
  \draw<1->[fill] (10.5,-10.) rectangle (10.45,-9.);
  \node<1-> at (10.45,-8.8) {\tiny $I$};
  \draw<1->[fill] (19.5,-10.) rectangle (19.45,-9.);
  \node<1-> at (19.45,-8.8) {\tiny $G$};
  
  \draw<1->[->] (10.5,-9.5) -- (S1);
  \draw<1->[->] (S1) edge [] (S2);
  \draw<1->[->] (S2) edge [] (S3);
  \draw<1->[->] (S3) edge [] (S4);
  \draw<1->[->] (S4) edge [] (S5);
  \draw<1->[->] (S5) edge [] (S6);
  \draw<1->[->] (S6) edge [] (S7);
  \draw<1->[->] (S7) edge [] (S8);
  \draw<1->[->,shorten >=.5mm] (S8) edge [] (19.5,-9.5);
  
  % new: 1-3 was: 2-
  \node<1-3> at ($(11,-10.5)+1/2*(nodeDifferenceH)$) (label-S0) {\tiny $s_{1}$};
  \foreach \x in {1,...,8} {
    \edef\xPlusOne{\x}
    \pgfmathparse{int(\xPlusOne+1)}
    \edef\xPlusOne{\pgfmathresult}
    \node<1-3> at ($(label-S0)+\x*(nodeDifferenceH)$)   {\tiny $s_{\xPlusOne}$};
  }
  \node<1-3> at ($(label-S0)-(nodeDifferenceH)$) {\tiny $s_{I}=s_0$};
  
  % new: 1-2 was: 3-
  \node<1-2> at ($(11.5,-9.25)-1/2*(nodeDifferenceH)$) (label-A1) {\tiny $a_{1}$};
  \foreach \x in {1,...,8} {
    \edef\xPlusOne{\x}
    \pgfmathparse{int(\xPlusOne+1)}
    \edef\xPlusOne{\pgfmathresult}
    \node<1-2> at ($(label-A1)+\x*(nodeDifferenceH)$)   {\tiny $a_{\xPlusOne}$};
  }
  
  \foreach \x in {11.5,...,19.5} {
    \edef\upperIndex{\x}
    \pgfmathparse{int(\upperIndex-10.5)}
    \edef\upperIndex{\pgfmathresult}
    \node<3-> at ($(\x,-9.25)-1/2*(nodeDifferenceH)$) (label-A\upperIndex) {\tiny $a_{1}^{\upperIndex}$};
    \foreach \y in {1,...,4} {
      \edef\yPlusOne{\y}
      \pgfmathparse{int(\yPlusOne+1)}
      \edef\yPlusOne{\pgfmathresult}
      \node<3-> at ($(label-A\upperIndex)+\y*(nodeDifferenceV)$)    {\tiny $a_{\yPlusOne}^{\upperIndex}$};
    }
  }
  
  \foreach \x in {10,...,19} {
    \edef\upperIndex{\x}
    \pgfmathparse{int(\upperIndex-10)}
    \edef\upperIndex{\pgfmathresult}
    \node<4-> at ($(\x,-10.5)+1/2*(nodeDifferenceH)$) (label-V\upperIndex) {\tiny $v_{1}^{\upperIndex}$};
     \foreach \y in {1,...,3} {
      \edef\yPlusOne{\y}
      \pgfmathparse{int(\yPlusOne+1)}
      \edef\yPlusOne{\pgfmathresult}
      \node<4-> at ($(label-V\upperIndex)-\y*(nodeDifferenceV)$)    {\tiny $v_{\yPlusOne}^{\upperIndex}$};
    }
  }
\end{tikzpicture}

	\end{center}
	\visible<2->{We only need two types of decision variables!}
	\begin{enumerate}
		\item<3-> $a_{i}^t$ -- Action $i$ is executed at time $t$.
		\item<4-> $v_{i}^t$ -- State variable $i$ is true at time $t$.
	\end{enumerate}
\end{frame}




\begin{frame}<2->{Overall Formula}
	\vspace{-0.1cm}
	\begin{center}
      \begin{tikzpicture}
  \coordinate (nodeDifferenceH) at (1,0);
  \coordinate (nodeDifferenceV) at (0,.35);
  
  \node at (21,-12) {};
  \node at (10,-7.5) {};
  
  \node<1->[S] (S1) at (11.5,-9.5) {};
  \node<1->[S] (S2) at ($(S1)+1*(nodeDifferenceH)$) {};
  \node<1->[S] (S3) at ($(S1)+2*(nodeDifferenceH)$) {};
  \node<1->[S] (S4) at ($(S1)+3*(nodeDifferenceH)$) {};
  \node<1->[S] (S5) at ($(S1)+4*(nodeDifferenceH)$) {};
  \node<1->[S] (S6) at ($(S1)+5*(nodeDifferenceH)$) {};
  \node<1->[S] (S7) at ($(S1)+6*(nodeDifferenceH)$) {};
  \node<1->[S] (S8) at ($(S1)+7*(nodeDifferenceH)$) {};
  \draw<1->[fill] (10.5,-10.) rectangle (10.45,-9.);
  \node<1-> at (10.45,-8.8) {\tiny $I$};
  \draw<1->[fill] (19.5,-10.) rectangle (19.45,-9.);
  \node<1-> at (19.45,-8.8) {\tiny $G$};
  
  \draw<1->[->] (10.5,-9.5) -- (S1);
  \draw<1->[->] (S1) edge [] (S2);
  \draw<1->[->] (S2) edge [] (S3);
  \draw<1->[->] (S3) edge [] (S4);
  \draw<1->[->] (S4) edge [] (S5);
  \draw<1->[->] (S5) edge [] (S6);
  \draw<1->[->] (S6) edge [] (S7);
  \draw<1->[->] (S7) edge [] (S8);
  \draw<1->[->,shorten >=.5mm] (S8) edge [] (19.5,-9.5);
  
  % new: 1-3 was: 2-
  \node<1-3> at ($(11,-10.5)+1/2*(nodeDifferenceH)$) (label-S0) {\tiny $s_{1}$};
  \foreach \x in {1,...,8} {
    \edef\xPlusOne{\x}
    \pgfmathparse{int(\xPlusOne+1)}
    \edef\xPlusOne{\pgfmathresult}
    \node<1-3> at ($(label-S0)+\x*(nodeDifferenceH)$)   {\tiny $s_{\xPlusOne}$};
  }
  \node<1-3> at ($(label-S0)-(nodeDifferenceH)$) {\tiny $s_{I}=s_0$};
  
  % new: 1-2 was: 3-
  \node<1-2> at ($(11.5,-9.25)-1/2*(nodeDifferenceH)$) (label-A1) {\tiny $a_{1}$};
  \foreach \x in {1,...,8} {
    \edef\xPlusOne{\x}
    \pgfmathparse{int(\xPlusOne+1)}
    \edef\xPlusOne{\pgfmathresult}
    \node<1-2> at ($(label-A1)+\x*(nodeDifferenceH)$)   {\tiny $a_{\xPlusOne}$};
  }
  
  \foreach \x in {11.5,...,19.5} {
    \edef\upperIndex{\x}
    \pgfmathparse{int(\upperIndex-10.5)}
    \edef\upperIndex{\pgfmathresult}
    \node<3-> at ($(\x,-9.25)-1/2*(nodeDifferenceH)$) (label-A\upperIndex) {\tiny $a_{1}^{\upperIndex}$};
    \foreach \y in {1,...,4} {
      \edef\yPlusOne{\y}
      \pgfmathparse{int(\yPlusOne+1)}
      \edef\yPlusOne{\pgfmathresult}
      \node<3-> at ($(label-A\upperIndex)+\y*(nodeDifferenceV)$)    {\tiny $a_{\yPlusOne}^{\upperIndex}$};
    }
  }
  
  \foreach \x in {10,...,19} {
    \edef\upperIndex{\x}
    \pgfmathparse{int(\upperIndex-10)}
    \edef\upperIndex{\pgfmathresult}
    \node<4-> at ($(\x,-10.5)+1/2*(nodeDifferenceH)$) (label-V\upperIndex) {\tiny $v_{1}^{\upperIndex}$};
     \foreach \y in {1,...,3} {
      \edef\yPlusOne{\y}
      \pgfmathparse{int(\yPlusOne+1)}
      \edef\yPlusOne{\pgfmathresult}
      \node<4-> at ($(label-V\upperIndex)-\y*(nodeDifferenceV)$)    {\tiny $v_{\yPlusOne}^{\upperIndex}$};
    }
  }
\end{tikzpicture}

	\end{center}
	\vspace{-.5cm}
	\visible<3->{Constraints to check:}
	\begin{itemize}
		\item<3-> Correctly applying actions at each time step ($\tau$).
		\item<4-> $I$ and $G$ must be respected.
	\end{itemize}
	\vspace{-.1cm}
	\begin{align*}
		\visible<3->{\mathcal F =} \visible<3->{\bigwedge_{t=0}^{k-1} \tau(t)} \visible<4->{\wedge}
		\onslide<4->{\bigwedge_{v_i \in I} v_i^0 \wedge \bigwedge_{v_i \in V \setminus I} \hspace{-0.2cm}\neg v_i^0 \wedge \bigwedge_{v_i \in G} v_i^{k}}\visible<3->{\quad \text{here: $k=9$}}
	\end{align*}
\end{frame}



\begin{frame}{Classical Planning via SAT}
  \begin{center}
  \begin{tikzpicture}
    \node[S] (S1) at (-2,0) {};
    \node[above of=S1,yshift=-.5cm] {$s$};
    \node[S] (S2) at (0,0)  {};
    \node[above of=S2,yshift=-.5cm] {$s'$};
    \draw[->] (S1) --  node[midway,above] {$a$} (S2);
  \end{tikzpicture}
  \end{center}
  Constraints to check by $\tau(t)$:
  \begin{itemize}[<+(1)->]
	\item[$F_1$] Preconditions must hold (in $s$).
	\item[$F_2$] Effects must occur (in $s'$).
    \item[$F_3$] Unaffected state variables stay unchanged.
    \item[$F_4$] At most one action per timestep.
    \item[$F_5$] At least one action per timestep. \visible<+->{Necessary?} \visible<+->{\textbf{No.}}
  \end{itemize}
\end{frame}






\begin{frame}{Classical Planning via SAT}
 	\begin{itemize}
		\item Preconditions must hold:
\visible<2->{\[F_1 = \bigwedge_{a \in A} a^{t+1} \rightarrow \bigwedge_{v \in pre(a)} v^t\]}
		\item Effects must occur:
			\begin{align*}\visible<3->{
					F_2 = &\left[\bigwedge_{a \in A} a^{t+1} \rightarrow	\bigwedge_{v \in add(a)} v^{t+1}\right]  \hspace{0.75cm} \wedge \\
					&\left[\bigwedge_{a \in A} a^{t+1} \rightarrow \bigwedge_{v \in del(a)} \neg v^{t+1}\right]}
			\end{align*}
	\end{itemize}
\end{frame}






\begin{frame}{Classical Planning via SAT}
 	\begin{itemize}
	\item Variables not affected by the executed action must stay the same.
		\begin{itemize}
			\item<2->[$\rightarrow$] Frame Problem!
		\end{itemize}
			\visible<3->{\begin{align*}F_3 = &\bigwedge_{v \in V} \left[(\neg v^{t} \wedge v^{t+1}) \rightarrow \hspace{-0.2cm}\bigvee_{a \in A \text{ with } v \in add(a)} \hspace{-0.8cm} a^{t+1} \hspace{0.4cm}\right] \wedge \\
				&\bigwedge_{v \in V}  \left[(v^{t} \wedge \neg v^{t+1}) \rightarrow \hspace{-0.2cm}\bigvee_{a \in A \text{ with } v \in del(a)} \hspace{-0.8cm} a^{t+1} \hspace{0.4cm}\right]
		\end{align*}}
		\item Only one action at a time\footnote{Various encodings exist. Typical are pairwise and bitwise. See \url{https://pysathq.github.io/docs/html/api/card.html} for a handy reference}:
			\visible<4->{\[F_4 = \sum_{a \in A} a^t \leq 1\]}
	\end{itemize}
\end{frame}


\begin{frame}{Extended Expressiveness}
 	\begin{itemize}
		\item Conditional Effects
 		\begin{itemize}
			\item at least since Madagascar, Rintanen, Heljanko, Niemelä, JAIR'06
		\end{itemize}
		\vspace{0.2cm}
		\pause
		\item Axioms / Derived Predicates
 		\begin{itemize}
			\item AxSAT, Behnke, Speck, Gnad, JELIA'25
		\end{itemize}
	\end{itemize}
\end{frame}


\begin{frame}{Other Base Encodings}
 	\begin{itemize}
		\item Partial-Order Causal-Link (POCL) encodings
 		\begin{itemize}
			\item Kautz, McAllester, Selman, KR'96\\ Depending on implementation cubic or quadratic in plan length
		\end{itemize}
		\vspace{0.2cm}
		\pause
		\item Encoding for SAS+ tasks:
 		\begin{itemize}
			\item SASE, Huang, Chen, Zhang, JAIR'12\\ quadratic (vars) and cubic (clauses) in $|D(v)|$
		\end{itemize}
		\vspace{0.2cm}
		\pause
		\item Encodings of Lifted Planning:
 		\begin{itemize}
			\item LiSAT, based on POCL, Höller\&Behnke, ICAPS'22
				\\ quadratic in plan length
			\item based on QBF, Shaik \& van de Pol, ICAPS'22
				\\ quadratic in plan length
		\end{itemize}
	\end{itemize}
\end{frame}







%
%
%
%\subsection{At-most-one}
%
%
%\newcommand{\TwoRowTab}[2]{\begin{tabular}{c}\footnotesize \ensuremath{#1}\\[-.25em]\footnotesize \ensuremath{#2}\end{tabular}}
%\begin{frame}{At-most-one}
%	Given a set of decision variables $X = \{x_1, \dots, x_{|X|}\}$. Find a set of clauses that, if satisfied, will ensure that at most one $x \in X$ is true.\\[\baselineskip]
%	\visible<2->{Naive encoding:}
%	\visible<2->{\[
%			\bigwedge_{x_1 \in X} \bigwedge_{x_2 \in X \setminus \{x_1\}} \underbrace{\neg x_1 \vee \neg x_2}_{\TwoRowTab{(x_1\Rightarrow \neg x_2)\ \wedge}{(x_2\Rightarrow \neg x_1)}}
%	\]}
%\end{frame}
%
%
%
%
%
%
%
%\begin{frame}{At-most-one}
%	Idea: Introduce new variables!
%	\pause
%	\begin{align*}
%		f_i &\text{ -- from index $i$ on all $x_i$ will be false}\\
%		&\text{i.e. it is \textbf{f}orbidden to use any $x_i$ after $i$}
%	\end{align*}
%	\pause
%	 Sequential encoding:
%	\begin{align*}
%		&\bigwedge_{i=1}^{|X|-1} \underbrace{\neg x_i \vee f_i}_{x_i \Rightarrow f_i} 
%		&\bigwedge_{i=2}^{|X|-1} \underbrace{\neg f_{i-1} \vee f_{i}}_{f_{i-1} \Rightarrow f_{i}}
%	\end{align*}
%	\begin{align*}
%		&\bigwedge_{i=1}^{|X|} \underbrace{\neg x_i \vee \neg f_{i-1}}_{\TwoRowTab{(x_i \Rightarrow \neg f_{i-1})\ \wedge}{(f_{i-1} \Rightarrow \neg x_i)}}
%	\end{align*}
%	%\pause
%\end{frame}
%
%
%
%
%
%
%\begin{frame}{At-most-one}
%	Maybe this is a bit much ...
%	\pause
%	\[n_i \text{ -- bit $i$ (0-index) of a $\lceil\log(|X|)\rceil$-digit binary number if one}\]
%	\pause
%	Binary encoding:
%	\begin{align*}
%		\neg x_i \vee n_j \quad \quad \text{ if } \frac{i}{2^{j}} \mod 2 = 1\\
%		\neg x_i \vee \neg n_j \quad \quad \text{ if } \frac{i}{2^{j}} \mod 2 = 0\\
%	\end{align*}
%\end{frame}
%
%
%
%
%%\renewcommand{\citeA}{1}
%
%\begin{frame}{Different AMO Implementations}
%	Overview\footnote{Frisch and Giannaros; SAT Encodings of the At-Most-k Constraint -- Some Old, Some New, Some Fast, Some Slow; 2010}
%	
%	\begin{center}
%	\begin{tabular}{l|l|l|l|l}
%		encoding & \#clauses & \#new variables \\ \hline \hline
%		binomial & $n^{2^{\phantom{1}}}$ & 0 \\\hline
%		binary & $n \log n$ & $\log n$\\\hline
%		sequential & $3n$ & $n$\\\hline
%		commander & $\frac{7}{2}n^{\phantom{1^{\phantom{1}}}}$ & $\frac{n}{2}$\\\hline
%		product & $2(n + n^{\frac{^{\phantom{1}}1^{\phantom{1}}}{m+1}})$ & $2n^{\frac{1}{2}}$\\
%	\end{tabular}\ \\[1em]
%	where $n$ is the number of atoms, i.e., $|X|$
%    \end{center}
%
%
%\end{frame}
%
%
%%\againframe{BoundIteration}
%
%
%\begin{frame}{Classical Planning via SAT}
%	\begin{center}
%		There are \textbf{a lot} of improvements to this formula.
%	\end{center}
%	\begin{itemize}[<+(1)->]
%		\item Invariants.
%		\item $\forall$-step semantics.
%		\item $\exists$-step semantics.
%	\end{itemize}
%\end{frame}

\section{Invariants}


\begin{frame}{What are Invariants?}
\begin{center}
	Is there \textbf{anything} we know about states in a planning problem?
\end{center}

\pause
\begin{definition}[Invariant]
	An invariant $\mathcal I$ is a formula over the state variables such that for all states $s$ reachable from $s_I$ it holds $s \models \mathcal I$.
\end{definition}

\pause

There are known methods for computing invariants \\ \qquad \qquad (e.g. Rintanen, AAAI'00 or Alcázar \&  Torralba, ICAPS'15).\\
\pause
The ones that interest us here are binary disjunctive invariants:
\[\ell_1 \vee \ell_2\]
\pause
	What to do with an invariant $\ell_1 \vee \ell_2$?\pause
	Add it to every timestep $t$ as $\ell_1^t \vee \ell_2^t$.\\[\baselineskip]
\pause These clauses are probably helpful for SAT-based planner, Rintanen, KR'08
\end{frame}


\section{$\forall$-step}

\begin{frame}{Linear Plans are Bad!}
Consider the following (single) planning problem:
\begin{center}\begin{tikzpicture}[scale=1.0]
    \tikzstyle{N}=[draw,circle,minimum size=4pt,inner sep=0pt]
	
	\node[N,blue] (A1) at (0,0) {C};
	\node[N] (A2) at (2,0) {B}; \node at (2.5,0) {\faArchive};
	\node[N] (A3) at (1,2) {A}; \node at (0.5,2) {\faTruck};
	\draw (A1) -- (A2);
	\draw (A2) -- (A3);
	\draw (A3) -- (A1);
	
	\node[N] (B1) at (5,0) {F}; \node at (4.5,0) {\faTruck};
	\node[N,blue] (B2) at (7,0) {E};
	\node[N] (B3) at (6,2) {D}; \node at (6.5,2) {\faArchive};
	\draw (B1) -- (B2);
	\draw (B2) -- (B3);
	\draw (B3) -- (B1);
 \end{tikzpicture}\\[\baselineskip]
\pause
\scalebox{0.8}{$\text{drive}(A,B), \text{load}(B), \text{drive}(B,C), \text{unload}(C), \text{drive}(F,D), \text{load}(D), \text{drive}(D,E), \text{unload}(E)$}\\[\baselineskip]
\pause
	\scalebox{0.8}{\begin{tabular}{l l l l}
		drive$(A,B)$ & load$(B)$ & drive$(B,C)$ & unload$(C)$\\
		drive$(F,D)$ & load$(D)$ & drive$(D,E)$ & unload$(E)$
	\end{tabular}}
\end{center}
\end{frame}





\begin{frame}{$\forall$-step [Kautz\&Selman, AAAI'96]}
\begin{center}
	Allow parallel execution of actions.\qquad But when?
\end{center}
\pause
\begin{itemize}[<+->]
    \item Let $\mathcal A$ be some set of actions.
	\item Parallel execution of $\mathcal A$ is safe, if all ($\forall$) linearisations of $\mathcal A$ are executable.% (Note the similarity to POCL planning.)
	\item Necessary conditions:
	\begin{itemize}
		\item All actions are executable in the previous state as all could be the first.
		\item No action can have a delete-effect that is a precondition of another action, i.e., $\forall a_1 \neq a_2 \in \mathcal A: del(a_1) \cap prec(a_2) = \emptyset$, as $a_1$ can occur directly before $a_2$.
	\end{itemize}
	\item Sufficient conditions: \visible<7->{Necessary conditions are already sufficient.}
\end{itemize}
\vspace{0.3cm}
\visible<8->{Further requirements for nice encoding?}\\
\visible<9->{The resulting state must unique!}\\
\visible<10->{We forbid two actions $a_1,a_2$ with $del(a_1) \cap add(a_2) \neq \emptyset$ to be executed in parallel.}\\
\visible<11->{We don't have to model this: $a_1^t \implies \neg v^t \wedge a_2^t \implies v^t$ imples $\neg a_1^t \lor a_2^t$.}
\end{frame}






\begin{frame}{Encoding $\forall$-step}
\hspace*{-.3cm}\Wider{1.04}{
    Remove the at-most-one constraints and add:\\[-2.5em]\pause

	\begin{align*}
		a_1^t \rightarrow \neg a_2^t \quad \quad \forall a_1,a_2 \in A\text{ with } del(a_1) \cap pre(a_2) \neq \emptyset
	\end{align*}\ \\[-2em]
	\centering $\rightarrow$ quadratic effort.\pause\\
	\vspace{1cm}Is this the best we can do? \pause \alert{No!}
	}
\end{frame}



\begin{frame}{Encoding Interference}
\textbf{Idea 1}: switch from a action-centric to a state variable-centric view.\\
\qquad \visible<2->{For every $v \in V$: if $v \in add(a_1)$ and $v \in del(a_2)$ add $a_1^t \rightarrow \neg a_2^t$}\\
\visible<3->{\textbf{Idea 2}: if one action with $v \in del(a_2)$ is forbidden, so are all others.}\\
\visible<4->{\textbf{Idea 3}: express this with additional variables!}\\
\visible<5->{The only problem is that an operation must not disable itself.}\\
\vspace{0.3cm}
\visible<6->{Arrange the actions with $v \in pre(a) \cup del(a)$ as a sequence $S$.}\\[0.2cm]
\visible<7->{\begin{tabular}{lllllllllllllll}
$a_1$ & $a_2$ & $a_3$ & $a_4$ & $a_5$ & $a_6$ & $a_7$ & $a_8$ & $a_9$\\
\visible<8->{E &E&E& &E& & & E&E}\\
\visible<8->{R &R&  &R&R&R&R&&R&}\\
\end{tabular}
}
\vspace{0.4cm}
    \begin{itemize}
      \item<8-> $E_v$ -- subsequence of $S$ with $v \in del(a)$ (\textbf{E}rasing)
      \item<8-> $R_v$ -- subsequence of $S$ with $v \in pre(a)$ (\textbf{R}equiring)
    \end{itemize}
\end{frame}

\begin{frame}{Chains}
\begin{tikzpicture}
    \node<1->[] at (0,-0.0) {E};
    \node<1->[] at (1,-0.0) {E};
    \node<1->[] at (2,-0.0) {E};
    \node<1->[] at (4,-0.0) {E};
    \node<1->[] at (7,-0.0) {E};
    \node<1->[] at (8,-0.0) {E};

    \node<1->[] at (0,3.0) {R};
    \node<1->[] at (1,3.0) {R};
    \node<1->[] at (3,3.0) {R};
    \node<1->[] at (4,3.0) {R};
    \node<1->[] at (5,3.0) {R};
    \node<1->[] at (6,3.0) {R};
    \node<1->[] at (8,3.0) {R};

    \node<1->[P,label={\tiny\ensuremath{a_1}}] (A1) at (0,1.5) {};
    \node<1->[P,label={\tiny\ensuremath{a_2}}] (A2) at (1,1.5) {};
    \node<1->[P,label={\tiny\ensuremath{a_3}}] (A3) at (2,1.5) {};
    \node<1->[P,label={\tiny\ensuremath{a_4}}] (A4) at (3,1.5) {};
    \node<1->[P,label={\tiny\ensuremath{a_5}}] (A5) at (4,1.5) {};
    \node<1->[P,label={\tiny\ensuremath{a_6}}] (A6) at (5,1.5) {};
    \node<1->[P,label={\tiny\ensuremath{a_7}}] (A7) at (6,1.5) {};
    \node<1->[P,label={\tiny\ensuremath{a_8}}] (A8) at (7,1.5) {};
    \node<1->[P,label={\tiny\ensuremath{a_9}}] (A9) at (8,1.5) {};

	\node<2->[S] (E1) at (0.5,2.5) {};
	\node<2->[S] (E2) at (2.5,2.5) {};
	\node<2->[S] (E3) at (3.5,2.5) {};
	\node<2->[S] (E4) at (4.5,2.5) {};
	\node<2->[S] (E5) at (5.5,2.5) {};
	\node<2->[S] (E6) at (7.5,2.5) {};
	\draw<3->[->] (A1) -- (E1) {};
	\draw<3->[->] (A2) -- (E2) {};
	\draw<3->[->] (A3) -- (E2) {};
	\draw<3->[->] (A5) -- (E4) {};
	\draw<3->[->] (A8) -- (E6) {};
	\draw<4->[->] (E1) -- (E2) {};
	\draw<4->[->] (E2) -- (E3) {};
	\draw<4->[->] (E3) -- (E4) {};
	\draw<4->[->] (E4) -- (E5) {};
	\draw<4->[->] (E5) -- (E6) {};
	\draw<5->[->,red,thick] (E1) -- (A2);
	\draw<5->[->,red,thick] (E2) -- (A4);
	\draw<5->[->,red,thick] (E3) -- (A5);
	\draw<5->[->,red,thick] (E4) -- (A6);
	\draw<5->[->,red,thick] (E5) -- (A7);
	\draw<5->[->,red,thick] (E6) -- (A9);



	\node<6->[S] (F1) at (0.5,0.5) {};
	\node<6->[S] (F2) at (1.5,0.5) {};
	\node<6->[S] (F3) at (3.5,0.5) {};
	\node<6->[S] (F4) at (4.5,0.5) {};
	\node<6->[S] (F5) at (5.5,0.5) {};
	\node<6->[S] (F6) at (6.5,0.5) {};
	\draw<6->[->] (A9) -- (F6) {};
	\draw<6->[->] (A8) -- (F6) {};
	\draw<6->[->] (A5) -- (F3) {};
	\draw<6->[->] (A3) -- (F2) {};
	\draw<6->[->] (A2) -- (F1) {};
	\draw<6->[->] (F2) -- (F1) {};
	\draw<6->[->] (F3) -- (F2) {};
	\draw<6->[->] (F4) -- (F3) {};
	\draw<6->[->] (F5) -- (F4) {};
	\draw<6->[->] (F6) -- (F5) {};
	\draw<6->[->,red,thick] (F1) -- (A1);
	\draw<6->[->,red,thick] (F2) -- (A2);
	\draw<6->[->,red,thick] (F3) -- (A4);
	\draw<6->[->,red,thick] (F4) -- (A5);
	\draw<6->[->,red,thick] (F5) -- (A6);
	\draw<6->[->,red,thick] (F6) -- (A7);

    \node<7->[P,blue] at (A3) (2,1.5) {};
	\node<7->[S,blue] at (E2) {};
	\node<7->[S,blue] at (E3) {};
	\node<7->[S,blue] at (E4) {};
	\node<7->[S,blue] at (E5) {};
	\node<7->[S,blue] at (E6) {};
	\node<7->[S,blue] at (F1) {};
	\node<7->[S,blue] at (F2) {};


    %\node<1->[P,label={[below,label distance =-0.3cm]\tiny\ensuremath{a_2}}] (A2) at (1,1) {};
    %\node<1->[P,label={\tiny\ensuremath{a_5}}] (A3) at (1,2) {};
    %\node<1->[P,label={\tiny\ensuremath{a_4}}] (A5) at (2,2) {};
    %\node<1->[P,label={[below,label distance =-0.3cm]\tiny\ensuremath{a_3}}] (A4) at (2,1) {};
    %\draw<1->[->] (A1) -- (A2);
    %\draw<1->[->] (A3) -- (A1);
    %\draw<3->[->,red] (A3) -- (A1);
    %\draw<1->[->] (A3) -- (A2);
    %\draw<3->[->,red] (A3) -- (A2);
    %\draw<1->[->] (A2) -- (A4);
    %\draw<1->[->] (A4) -- (A5);
    %\draw<1->[->] (A5) -- (A3);

\end{tikzpicture}
      \visible<8->{\begin{align*}
        chain(E&,R) = \\
		\bigwedge &\{a^i \rightarrow \mathtt{f}^j \mid i < j, a_i \in E, a_j \in R, \{a_{i+1}, \dots, a_{j-1}\} \cap R = \emptyset\} \cup {}\\
        &\{\mathtt{f}^i \rightarrow \mathtt{f}^j \mid i < j, \{a_i, a_j\} \in R, \{a_{i+1}, \dots, a_{j-1}\} \cap R = \emptyset\} \cup {}\\
      &\{\mathtt{f}^i \rightarrow \neg a_i \mid a_i \in R\}
      \end{align*}}
\visible<9->{Two chains for every $v \in V$ with \alert{fresh} decision variables $\texttt{f}^i$.}
\end{frame}





\section{$\exists$-step}



\begin{frame}{Parallel Plans are (Still) Bad!}
(Re-)Consider the following (single) planning problem:
\begin{center}\begin{tikzpicture}[scale=1.0]
    \tikzstyle{N}=[draw,circle,minimum size=4pt,inner sep=0pt]
  
  \node[N,blue] (A1) at (0,0) {C};
  \node[N] (A2) at (2,0) {B}; \node at (2.5,0) {\faArchive};
  \node[N] (A3) at (1,2) {A}; \node at (0.5,2) {\faTruck};
  \draw (A1) -- (A2);
  \draw (A2) -- (A3);
  \draw (A3) -- (A1);

  \node[N] (B1) at (5,0) {F}; \node at (4.5,0) {\faTruck};
  \node[N,blue] (B2) at (7,0) {E};
  \node[N] (B3) at (6,2) {D}; \node at (6.5,2) {\faArchive};
  \draw (B1) -- (B2);
  \draw (B2) -- (B3);
  \draw (B3) -- (B1);
 \end{tikzpicture}\\[\baselineskip]
  \scalebox{0.8}{\begin{tabular}{l l l l}
    drive$(A,B)$ & load$(B)$ & drive$(B,C)$ & unload$(C)$\\
    drive$(F,D)$ & load$(D)$ & drive$(D,E)$ & unload$(E)$
  \end{tabular}} \\[\baselineskip]
\pause
  \scalebox{0.8}{\begin{tabular}{l l l}
    drive$(A,B)$ & load$(B)$ & unload$(C)$\\
           & drive$(B,C)$ &\\
    drive$(F,D)$ & load$(D)$ & unload$(E)$\\
           & drive$(D,E)$ &
  \end{tabular}}
\end{center}
\end{frame}





\begin{frame}{What Kind of Parallelism do we Look for?}
\hspace*{-.5cm}\Wider{1.04}{
\begin{itemize}
    \item<2-> Absolutely safe parallelism.
      \begin{itemize}
        \item<3-> All linearisations will always be executable and lead to the same state.
        \item<3-> $\forall$-step.
      \end{itemize}
    \item<4-> (Sometimes) Safe parallelism.
      \begin{itemize}
        \item<5-> At least one linearisation is executable and all executable linearisations lead to the same state.
        \item<5-> $\exists$-step.
      \end{itemize}
  \end{itemize}
  }
\end{frame}





\begin{frame}{$\exists$-step Parallelism}
  \begin{itemize}
    \item<1-> Given a set of actions $\mathcal A$. We call them $\exists$-step executable if a linearisation exists that is executable and all executable linearisations lead to the same state.
    \item<2-> How difficult to determine? \\
		\visible<3->{First part is $\mathbb{NP}$-complete. [Rintanen,Heljanko,Niemel\"a, JAIR'06]}
    \item<4-> How to encode?
	\item<5-> Results in the Kautz\&Selman encoding ...
  \end{itemize}
\end{frame}





\begin{frame}{Approximating True $\exists$-step Semantics}
	\begin{itemize}[<+->]
		\item As a good planning researcher: approximate $\exists$-step semantics.\\[0.3cm]
		\item Fix an ordering of action $\vec{\mathcal A}$.
		\item Executing $\mathcal B \subseteq \mathcal A$ is allowed in parallel, \textbf{iff} they can be executed in the order in $\vec{\mathcal A}$.\\[0.2cm]
	\end{itemize}

\visible<2->{
	\begin{center}
		$ \vec{\mathcal A} = 
		a_3\ 
		\alert<3->{a_7}\ 
		\alert<3->{a_2}\ 
		a_1\ 
		a_6\ 
		a_5\ 
		\alert<3->{a_4}\ 
		$\\
	\visible<3->{Execute: $\{a_2, a_4, a_7\}$}
	\end{center}
}
\end{frame}



\begin{frame}{Approximating True $\exists$-step Semantics}
\visible<1->{
	\begin{center}
		$\vec{\mathcal A} = 
		a_3\ 
		a_7\ 
		a_2\ 
		a_1\ 
		a_6\ 
		a_5\ 
		a_4\ 
		$\\
	\end{center}
}
How to encode that given subset of $\mathcal A$ is executable in the order of $\vec{\mathcal A}$?
\pause
	\begin{itemize}
		\item ``classical'' $\exists$-step [Rintanen,Heljanko,Niemel\"a, JAIR'06]
		\item relaxed $\exists$-step [Wehrle\&Rintanen, AJCAI'07]
		\item relaxed relaxed $\exists$-step [Balyo, ICTAI'13]
		\item reinforced encoding [Balyo, Barták, Trunda, PAIR'14]
	\end{itemize}
 \vfill
\end{frame}


\begin{frame}{``Classical'' $\exists$-step Semantics [Rintanen,Heljanko,Niemel\"a, JAIR'06]}
\visible<1->{
	\begin{center}
		$\vec{\mathcal A} = 
		a_3\ 
		a_7\ 
		a_2\ 
		a_1\ 
		a_6\ 
		a_5\ 
		a_4\ 
		$\\
	\end{center}
}
\pause
Conditions for executing $\mathcal B \subseteq \mathcal A$:
\pause
	\begin{itemize}[<+->]
		\item for all $a \in \mathcal B$: any precondition $p \in pre(a)$ is true in the current state
		\item no $a \in \mathcal B$ disables a $b \in \mathcal B$ which is later according to $\vec{\mathcal A}$.
		\item[$\rightarrow$] \textit{disables} means $a$ has deleting effect on precondition of $b$.
		\item no $a,b \in \mathcal B$ have contradictory effects
	\end{itemize}
\vspace{0.5cm}
\visible<7->{Encode \textbf{fact-based}: For every state variable $v$ consider the actions that
\begin{itemize}
	\item \textbf{E}rase $v$, i.e., $v \in del(a)$
	\item \textbf{R}equire $v$, i.e., $v \in pre(a)$
\end{itemize}
}

 \vfill
\end{frame}


\begin{frame}{Chains for ``Classical'' $\exists$-Step}
\begin{tikzpicture}
    \node<1->[] at (0,1.0) {E};
    \node<1->[] at (1,1.0) {E};
    \node<1->[] at (2,1.0) {E};
    \node<1->[] at (4,1.0) {E};
    \node<1->[] at (7,1.0) {E};
    \node<1->[] at (8,1.0) {E};

    \node<1->[] at (0,3.0) {R};
    \node<1->[] at (1,3.0) {R};
    \node<1->[] at (3,3.0) {R};
    \node<1->[] at (4,3.0) {R};
    \node<1->[] at (5,3.0) {R};
    \node<1->[] at (6,3.0) {R};
    \node<1->[] at (8,3.0) {R};

    \node<1->[P,label={\tiny\ensuremath{a_1}}] (A1) at (0,1.5) {};
    \node<1->[P,label={\tiny\ensuremath{a_2}}] (A2) at (1,1.5) {};
    \node<1->[P,label={\tiny\ensuremath{a_3}}] (A3) at (2,1.5) {};
    \node<1->[P,label={\tiny\ensuremath{a_4}}] (A4) at (3,1.5) {};
    \node<1->[P,label={\tiny\ensuremath{a_5}}] (A5) at (4,1.5) {};
    \node<1->[P,label={\tiny\ensuremath{a_6}}] (A6) at (5,1.5) {};
    \node<1->[P,label={\tiny\ensuremath{a_7}}] (A7) at (6,1.5) {};
    \node<1->[P,label={\tiny\ensuremath{a_8}}] (A8) at (7,1.5) {};
    \node<1->[P,label={\tiny\ensuremath{a_9}}] (A9) at (8,1.5) {};

	\node<2->[S] (E1) at (0.5,2.5) {};
	\node<2->[S] (E2) at (2.5,2.5) {};
	\node<2->[S] (E3) at (3.5,2.5) {};
	\node<2->[S] (E4) at (4.5,2.5) {};
	\node<2->[S] (E5) at (5.5,2.5) {};
	\node<2->[S] (E6) at (7.5,2.5) {};
	\draw<2->[->] (A1) -- (E1) {};
	\draw<2->[->] (A2) -- (E2) {};
	\draw<2->[->] (A3) -- (E2) {};
	\draw<2->[->] (A5) -- (E4) {};
	\draw<2->[->] (A8) -- (E6) {};
	\draw<2->[->] (E1) -- (E2) {};
	\draw<2->[->] (E2) -- (E3) {};
	\draw<2->[->] (E3) -- (E4) {};
	\draw<2->[->] (E4) -- (E5) {};
	\draw<2->[->] (E5) -- (E6) {};
	\draw<2->[->,red,thick] (E1) -- (A2);
	\draw<2->[->,red,thick] (E2) -- (A4);
	\draw<2->[->,red,thick] (E3) -- (A5);
	\draw<2->[->,red,thick] (E4) -- (A6);
	\draw<2->[->,red,thick] (E5) -- (A7);
	\draw<2->[->,red,thick] (E6) -- (A9);


    %\node<7->[P,blue] at (A3) (2,1.5) {};
	%\node<7->[S,blue] at (E2) {};
	%\node<7->[S,blue] at (E3) {};
	%\node<7->[S,blue] at (E4) {};
	%\node<7->[S,blue] at (E5) {};
	%\node<7->[S,blue] at (E6) {};
	%\node<7->[S,blue] at (F1) {};
	%\node<7->[S,blue] at (F2) {};


    %\node<1->[P,label={[below,label distance =-0.3cm]\tiny\ensuremath{a_2}}] (A2) at (1,1) {};
    %\node<1->[P,label={\tiny\ensuremath{a_5}}] (A3) at (1,2) {};
    %\node<1->[P,label={\tiny\ensuremath{a_4}}] (A5) at (2,2) {};
    %\node<1->[P,label={[below,label distance =-0.3cm]\tiny\ensuremath{a_3}}] (A4) at (2,1) {};
    %\draw<1->[->] (A1) -- (A2);
    %\draw<1->[->] (A3) -- (A1);
    %\draw<3->[->,red] (A3) -- (A1);
    %\draw<1->[->] (A3) -- (A2);
    %\draw<3->[->,red] (A3) -- (A2);
    %\draw<1->[->] (A2) -- (A4);
    %\draw<1->[->] (A4) -- (A5);
    %\draw<1->[->] (A5) -- (A3);

\end{tikzpicture}\\[0.3cm]
	\textbf{Exactly} the same encoding as for $\forall$-step parallelism, but we \textbf{omit} the second chain in the backwards direction.
\end{frame}


\begin{frame}{Relaxed $\exists$-step Semantics [Rintanen,Heljanko,Niemel\"a, JAIR'06]}
\visible<1->{
	\begin{center}
		$\vec{\mathcal A} = 
		a_3\ 
		a_7\ 
		a_2\ 
		a_1\ 
		a_6\ 
		a_5\ 
		a_4\ 
		$\\
	\end{center}
}
Conditions for executing $\mathcal B \subseteq \mathcal A$:
	\begin{itemize}
		\item \alert<2>{all $a \in \mathcal B$ are executable in the current state}
		\visible<3->{\textbf{or} are added by an action $b \in \mathcal B$ which is precedes $a$ according to $\vec{\mathcal A}$.}
		\item no $a \in \mathcal B$ disables a $b \in \mathcal B$ which is later according to $\vec{\mathcal A}$.
		\item no $a,b \in \mathcal B$ have contradictory effects
	\end{itemize}
\vspace{0.5cm}
\visible<4->{Encoding}
 	\begin{itemize}
		\item<4-> Preconditions must hold:
\visible<4->{\[F_1 = \bigwedge_{a \in A} a^{t+1} \rightarrow \bigwedge_{v \in pre(a)} \left(v^t \lor \bigvee_{b \prec_{\vec{\mathcal A}} a, v \in add(b)} b^{t} \right)\]}
 	\end{itemize}

 \vfill
\end{frame}

\renewcommand<>{\sout}[1]{\alt#2{\beameroriginal{\sout}{#1}}{#1}}

\begin{frame}{Relaxed Relaxed $\exists$-step Semantics [Balyo, ICTAI'13]}
\visible<1->{
	\begin{center}
		$\vec{\mathcal A} = 
		a_3\ 
		a_7\ 
		a_2\ 
		a_1\ 
		a_6\ 
		a_5\ 
		a_4\ 
		$\\
	\end{center}
}
Conditions for executing $\mathcal B \subseteq \mathcal A$:
	\begin{itemize}
		\item \sout<3->{all $a \in \mathcal B$ are executable in the current state or are added by an action $b \in \mathcal B$ which is precedes $a$ according to $\vec{\mathcal A}$.}
		\item \sout<3->{\alert<2>{no $a \in \mathcal B$ disables a $b \in \mathcal B$ which is later according to $\vec{\mathcal A}$.}}
		\item \sout<3->{\alert<2>{no $a,b \in \mathcal B$ have contradictory effects}}
		\item<4-> is executable in the order of $\vec{\mathcal A}$.
	\end{itemize}
\vspace{0.5cm}
\visible<5->{How do we encode this?}
 \vfill
\end{frame}

\begin{frame}{Chains for Relaxed Relaxed $\exists$-Step}
\begin{tikzpicture}
    \node<1->[] at (0,1.0) {E};
    \node<1->[] at (1,1.0) {E};
    \node<2->[] at (2,1.0) {S};
    \node<1->[] at (4,1.0) {E};
    \node<2->[] at (7,1.0) {S};
    \node<1->[] at (8,1.0) {E};

    \node<1->[] at (0,3.0) {R};
    \node<1->[] at (1,3.0) {R};
    \node<1->[] at (3,3.0) {R};
    \node<1->[] at (4,3.0) {R};
    \node<1->[] at (5,3.0) {R};
    \node<1->[] at (6,3.0) {R};
    \node<1->[] at (8,3.0) {R};


    \node<1->[P,label={\tiny\ensuremath{a_1}}] (A1) at (0,1.5) {};
    \node<1->[P,label={\tiny\ensuremath{a_2}}] (A2) at (1,1.5) {};
    \node<1->[P,label={\tiny\ensuremath{a_3}}] (A3) at (2,1.5) {};
    \node<1->[P,label={\tiny\ensuremath{a_4}}] (A4) at (3,1.5) {};
    \node<1->[P,label={\tiny\ensuremath{a_5}}] (A5) at (4,1.5) {};
    \node<1->[P,label={\tiny\ensuremath{a_6}}] (A6) at (5,1.5) {};
    \node<1->[P,label={\tiny\ensuremath{a_7}}] (A7) at (6,1.5) {};
    \node<1->[P,label={\tiny\ensuremath{a_8}}] (A8) at (7,1.5) {};
    \node<1->[P,label={\tiny\ensuremath{a_9}}] (A9) at (8,1.5) {};

	\node<1->[S] (E1) at (0.5,2.5) {};
	\node<1->[S] (E2) at (2.5,2.5) {};
	\node<1->[S] (E3) at (3.5,2.5) {};
	\node<1->[S] (E4) at (4.5,2.5) {};
	\node<1->[S] (E5) at (5.5,2.5) {};
	\node<1->[S] (E6) at (7.5,2.5) {};
	\draw<1->[->] (A1) -- (E1) {};
	\draw<1->[->] (A2) -- (E2) {};
	\draw<1->[->] (A5) -- (E4) {};
	\draw<1->[->] (E1) -- (E2) {};
	\draw<1->[->] (E2) -- (E3) {};
	\draw<1->[->] (E3) -- (E4) {};
	\draw<1->[->] (E4) -- (E5) {};
	\draw<1->[->] (E5) to node (E56) {} (E6) {};
	\draw<1->[->,red,thick] (E1) -- (A2);
	\draw<1->[->,red,thick] (E2) -- (A4);
	\draw<1->[->,red,thick] (E3) -- (A5);
	\draw<1->[->,red,thick] (E4) -- (A6);
	\draw<1->[->,red,thick] (E5) -- (A7);
	\draw<1->[->,red,thick] (E6) -- (A9);
	
	\draw<3->[->,blue,thick] (A8) -- ($(A8) + (0,1)$) {};
	\draw<3->[->,blue,thick] (A3) -- ($(A3) + (0,1)$) {};
	\draw<3->[->,blue,thick] (A3) -- ($(A3) + (0,0.7)$) {};


    %\node<7->[P,blue] at (A3) (2,1.5) {};
	%\node<7->[S,blue] at (E2) {};
	%\node<7->[S,blue] at (E3) {};
	%\node<7->[S,blue] at (E4) {};
	%\node<7->[S,blue] at (E5) {};
	%\node<7->[S,blue] at (E6) {};
	%\node<7->[S,blue] at (F1) {};
	%\node<7->[S,blue] at (F2) {};


    %\node<1->[P,label={[below,label distance =-0.3cm]\tiny\ensuremath{a_2}}] (A2) at (1,1) {};
    %\node<1->[P,label={\tiny\ensuremath{a_5}}] (A3) at (1,2) {};
    %\node<1->[P,label={\tiny\ensuremath{a_4}}] (A5) at (2,2) {};
    %\node<1->[P,label={[below,label distance =-0.3cm]\tiny\ensuremath{a_3}}] (A4) at (2,1) {};
    %\draw<1->[->] (A1) -- (A2);
    %\draw<1->[->] (A3) -- (A1);
    %\draw<3->[->,red] (A3) -- (A1);
    %\draw<1->[->] (A3) -- (A2);
    %\draw<3->[->,red] (A3) -- (A2);
    %\draw<1->[->] (A2) -- (A4);
    %\draw<1->[->] (A4) -- (A5);
    %\draw<1->[->] (A5) -- (A3);

\end{tikzpicture}\\[0.3cm]
	\visible<5->{
Encode this via extended chains:
	\begin{align*}
        ch&ain(E,R,S) = \\
		\bigwedge &\{a^i \land \hspace{-0.5cm} \bigwedge_{\neg a_k \in S, i < k < j} \hspace{-0.5cm} a_k \rightarrow \mathtt{f}^j \mid i < j, a_i \in E, a_j \in R, \{a_{i+1}, \dots, a_{j-1}\} \cap R = \emptyset\} \cup {}\\
        &\{\mathtt{f}^i \rightarrow \mathtt{f}^j \lor \hspace{-0.5cm} \bigvee_{a_k \in S, i < k < j} \hspace{-0.5cm} a_k  \mid i < j, \{a^i, a^j\} \in R, \{a_{i+1}, \dots, a_{j-1}\} \cap R = \emptyset\} \cup {}\\
      &\{\mathtt{f}^i \rightarrow \neg a_i \mid a_i \in R\}
      \end{align*}}
\end{frame}





\begin{frame}{Finding Good Action Orderings: Disabling Graph [Rintanen,Heljanko,Niemel\"a, JAIR'06]}
    \begin{minipage}{0.65\textwidth}
        \begin{itemize}[<+->]
            \item Analyse dependency between actions.
            \item Similar to $\forall$-step:
              \begin{itemize}[<+->]
                \item If $del(a) \cap pre(a') \neq \emptyset$, execute $a'$ before $a$.
                \item Ignore if $\mathcal I \cup pre(a) \cup pre(a')$ is inconsistent.
              \end{itemize}
          \end{itemize}    
    \end{minipage}
    \begin{minipage}{0.30\textwidth}
        \begin{tikzpicture}
            \node<1-> at (-0.3,.0) {};
            \node<1-> at (4,4) {};
            
            \node<4->[P,label={\tiny\ensuremath{a_1}}] (A1) at (0,1.5) {};
            \node<4->[P,label={[below,label distance =-0.3cm]\tiny\ensuremath{a_2}}] (A2) at (1,1) {};
            \node<4->[P,label={\tiny\ensuremath{a_3}}] (A3) at (1,2) {};
            \node<4->[P,label={[below,label distance =-0.3cm]\tiny\ensuremath{a_4}}] (A4) at (2,1) {};
            \node<4->[P,label={\tiny\ensuremath{a_5}}] (A5) at (3,3) {};
            \draw<4->[->] (A1) -- (A2);
            \draw<4->[->] (A1) -- (A3);
            \draw<4->[->] (A3) -- (A2);
            \draw<4->[->] (A2) -- (A4);
            \draw<4->[->] (A3) -- (A5);
            
            % the following 3 elements could either be shown
            % only in 4 or being crossed out in 5
            \draw<4->[->] (A4) -- (A3);
            \draw<4->[->] (A4) to[bend right] (A5);
            \draw<4->[->] (A5) to [bend right] (A4);
            \draw<5>[-] (1.75,2.25) -- (3.25,1.75);
            \draw<5>[-] (1.75,1.75) -- (3.25,2.25);
            \draw<5>[-] ($(.75,1.5)+(.5,.05)$) -- ($(1.25,1.5)+(.5,-.05)$);
            \draw<5>[-] ($(.75,1.5)+(.5,-.05)$) -- ($(1.25,1.5)+(.5,.05)$);
        \end{tikzpicture}
      \end{minipage}
\end{frame}






\begin{frame}{$\exists$-step [Rintanen,Heljanko,Niemel\"a'06]}
    \begin{minipage}{0.65\textwidth}
        \begin{itemize}[<+->]
            \item Disabling Graph: $a \rightarrow b$ iff after executing $a$ it may not be possible to execute $b$. 
            \item We can safely execute actions in reverse topological order.
            \item DG may not be acyclic.
            \item Guess an order in every SCC and order SCCs in reverse topological order.
            \item If executed in parallel, we will always execute actions in \textbf{this} order.
        \end{itemize}    
    \end{minipage}
    \begin{minipage}{0.30\textwidth}
        \begin{tikzpicture}
            \node<1->[P,label={\tiny\ensuremath{a_1}}] (A1) at (0,1.5) {};
            \node<1->[P,label={[below,label distance =-0.3cm]\tiny\ensuremath{a_2}}] (A2) at (1,1) {};
            \node<1->[P,label={\tiny\ensuremath{a_3}}] (A3) at (1,2) {};
            \node<1->[P,label={[below,label distance =-0.3cm]\tiny\ensuremath{a_4}}] (A4) at (2,1) {};
            \node<1->[P,label={\tiny\ensuremath{a_5}}] (A5) at (3,3) {};
            \draw<1->[->] (A1) -- (A2);
            \draw<1->[->] (A1) -- (A3);
            \draw<1->[->] (A3) -- (A2);
            \draw<1->[->] (A2) -- (A4);
            \draw<1->[->] (A3) -- (A5);
            \draw<3->[->] (A4) -- (A3);
        \end{tikzpicture}
        \begin{center}
            \visible<2>{$a_5,a_4,a_2,a_3,a_1$}\\
            \visible<4->{$(a_5),(a_2,a_3,a_4),(a_1)$}
        \end{center}
    \end{minipage}
\end{frame}


%\begin{frame}{$\exists$-step}
%    What do we have to assert inside the propositional formula?\\[0.2cm]
%    \begin{minipage}{0.65\textwidth}
%        \begin{itemize}
%            \item<2-> Parallel actions must result in a consistent state. \visible<3->{\checkmark}
%            \item<4-> Parallel actions must be executable.
%                \begin{enumerate}
%                    \item<5-> Actions must be applicable in the previous state.
%                    \item<6-> Reverse topological order of DG ensures that later actions are still applicable.
%                    \item<7-> In SCCs there might be edges opposite to the chosen order.
%                    \item<8-> SCC can be treated separately.
%                    \item<9-> If $a_2$ is executed, then $a_4$ must not.
%                    \item<10-> Enforced via \emph{chaines}.
%                \end{enumerate}
%        \end{itemize}
%    \end{minipage}
%    \begin{minipage}{0.30\textwidth}
%        \begin{tikzpicture}
%            \node at (0,0){};
%            \node at (3.5,3.5){};
%            \node<6->[P,label={\tiny\ensuremath{a_1}}] (A1) at (0,1.5) {};
%            \node<6->[P,label={[below,label distance =-0.3cm]\tiny\ensuremath{a_2}}] (A2) at (1,1) {};
%            \node<6->[P,label={\tiny\ensuremath{a_3}}] (A3) at (1,2) {};
%            \node<6->[P,label={[below,label distance =-0.3cm]\tiny\ensuremath{a_4}}] (A4) at (2,1) {};
%            \node<6->[P,label={\tiny\ensuremath{a_5}}] (A5) at (3,3) {};
%            \draw<6->[->] (A1) -- (A2);
%            \draw<6->[->] (A1) -- (A3);
%            \draw<6->[->] (A3) -- (A2);
%            \draw<6>[->]  (A2) -- (A4);
%            \draw<7->[->,color=red] (A2) -- (A4);
%            \draw<6->[->] (A3) -- (A5);
%            \draw<6->[->] (A4) -- (A3);
%        \end{tikzpicture}
%        \begin{center}
%            \visible<6>{$a_5,a_2,a_3,a_4,a_1$}
%            \visible<7->{$(a_5),(a_2,a_3,a_4),(a_1)$}
%        \end{center}
%    \end{minipage}
%\end{frame}
%
%
%
%
%\begin{frame}{$\exists$-step and Chains}
%%\vspace*{-.5cm}
%  \begin{center}
%  We are given an SCC and an ordering of its vertices.
%\begin{tikzpicture}
%    \node<1->[P,label={\tiny\ensuremath{a_1}}] (A1) at (0,1.5) {};
%    \node<1->[P,label={[below,label distance =-0.3cm]\tiny\ensuremath{a_2}}] (A2) at (1,1) {};
%    \node<1->[P,label={\tiny\ensuremath{a_5}}] (A3) at (1,2) {};
%    \node<1->[P,label={\tiny\ensuremath{a_4}}] (A5) at (2,2) {};
%    \node<1->[P,label={[below,label distance =-0.3cm]\tiny\ensuremath{a_3}}] (A4) at (2,1) {};
%    \draw<1->[->] (A1) -- (A2);
%    \draw<1->[->] (A3) -- (A1);
%    \draw<3->[->,red] (A3) -- (A1);
%    \draw<1->[->] (A3) -- (A2);
%    \draw<3->[->,red] (A3) -- (A2);
%    \draw<1->[->] (A2) -- (A4);
%    \draw<1->[->] (A4) -- (A5);
%    \draw<1->[->] (A5) -- (A3);
%
%    \node (pi) at (5,1.5) {$\pi = (a_5,a_4,a_3,a_2,a_1)$};
%    %\node[below of=pi,yshift=.5cm] {$\pi = (\pi^1,\pi^2,\pi^3,\pi^4,\pi^5)$};
%\end{tikzpicture}
%\end{center}
%\vspace*{-.25cm}
%  \begin{itemize}
%    \item<2-> We want choose an acyclic subsequence of $\pi$.
%    \item<3-> Do not choose both ends of a forward edge.
%    \item<4-> Iterate over causes of these edges: $v \in del(a_1) \cap pre(a_2)$
%    \begin{itemize}
%      \item<5-> $E_v$ -- subsequence of $\pi$ with $v \in del(a)$ (\textbf{E}rasing)
%      \item<6-> $R_v$ -- subsequence of $\pi$ with $v \in pre(a)$ (\textbf{R}equiring)
%    \end{itemize}
%	\item<7-> Add $chain(\pi,E_v,R_v)$ -- i.e.\ whenever an action erases $v$, we forbid any requiring action after it in $\pi$.
%  \end{itemize}
%\end{frame}


\section*{Summary}


\begin{frame}{Further Work}
  \footnotesize
  Improvements for classical planning:
  \begin{itemize}
    \item Parallel SAT search [Rintanen'04] [Rintanen,Heljanko,Niemel\"a'06].
    \item Specialised heuristics for SAT solvers [Rintanen'10a] [Rintanen'10b].
    \item Improved memory management [Rintanen'12].
    \item Incremental SAT-solving [Gocht\&Balyo'17]. 
	\item CEGAR-style parallelism [Froleyks,Balyo,Schreiber'19].
	\item Split representation of actions [Robinson,Gretton,Pham,Sattar'09].
	\item Optimal planning via MaxSAT [Robinson,Gretton,Pham,Sattar'10].
	\item Lifted Planning via SAT [Höller\&Behnke'22] or QBF [Shaik\&van de Pol'22].
  \end{itemize}
  \vspace{0.25cm}
  \pause
  Extensions to non-classical planning:
  \begin{itemize}
    \item LTL [Mattm\"uller\&Rintanen'07] [Behnke\&Biundo'18].
    \item Partial Observability [Pandey\&Rintanen'18].
    %\item Temporal Planning [Rintanen'17].
	\item HTN Planning [Behnke,H\"oller,Biundo'17'18].
	\item FOND [Geffner\&Geffner'18].
	\item POD [Pandey\&Rintanen'18] [Fadnis\&Rintanen'23].
	\item Derived Predicates [Behnke,Speck,Gnad'25].
  \end{itemize}
  %\begin{center}
  %  \visible<3->{$\rightarrow$ \url{https://users.aalto.fi/~rintanj1/satplan.html}}
  %\end{center}
\end{frame}
%\begin{frame}
%	Solving Problems via Translation into SAT:
%	\begin{itemize}
%		\item Problem transformation is a general and important concept in computer science.
%		\item SAT solvers are highly efficient and can be used to solve other difficult problems via transformation, even those in higher complexity classes with appropriate compilation.
%	\end{itemize}
%\vspace{0.3cm}
%
%	Translating Classical planning into SAT:
%	\begin{itemize}
%		\item Classical planning problems can be translated into SAT.
%		\item State-of-the-art improvements for this formula are based on:
%		\begin{itemize}
%			\item State invariants.
%			\item Parallelism ($\forall$-step, $\exists$-step).
%		\end{itemize}
%	\end{itemize}
%\end{frame}



%\subsection*{References}
%
%\begin{frame}
%\scriptsize
%\hspace*{2cm}
%	\scalebox{.99}{\begin{minipage}{0.9\textwidth}
%	\begin{itemize}
%		\item[Bylander'94] The Computational Complexity of Propositional STRIPS Planning.
%		\item[Kautz\&Selman'92] Planning as Satisfiability.
%		\item[Kautz\&Selman'96] Pushing the Envelope: Planning, Propositional Logic, and Stochastic Search.
%		\item[Rintanen'98] A Planning Algorithm not based on Directional Search.
%		\item[Rintanen'04] Evaluation Strategies for Planning as Satisfiability.
%		\item[Rintanen,Heljanko,Niemel\"a'06] Planning as Satisfiability: Parallel Plans and Algorithms for Plan Search.
%		\item[Wehrle\&Rintanen'07] Planning as Satisfiability with Relaxed $\exists$-Step Plans.
%		\item[Mattm\"uller\&Rintanen'07] Planning for Temporally Extended Goals as Propositional Satisfiability.
%		\item[Rintanen'10a] Heuristic Planning with SAT: Beyond Uninformed Depth-First Search.
%		\item[Rintanen'10b] Heuristics for Planning with SAT.
%		\item[Gocht\&Balyo'17] Accelerating SAT Based Planning with Incremental SAT Solving.
%		\item[Rintanen'17] Temporal Planning with Clock-Based SMT Encodings.
%		\item[Behnke\&Biundo'18] X and more Parallelism. Integrating LTL-Next into SAT-based Planning with Trajectory Constraints while Allowing for even more Parallelism.
%		\item[Randey\&Rintanen'18] Planning for Partial Observability by SAT and Graph Constraints.
%	\end{itemize}
%\end{minipage}}
%\end{frame}



\end{document}

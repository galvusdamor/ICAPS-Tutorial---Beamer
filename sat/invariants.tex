\section{Invariants}


\begin{frame}{What are Invariants?}
\begin{center}
	Is there \textbf{anything} we know about states in a planning problem?
\end{center}

\pause
\begin{definition}[Invariant]
	An invariant $\mathcal I$ is a formula over the state variables such that for all states $s$ reachable from $s_I$ it holds $s \models \mathcal I$.
\end{definition}

\pause

There are known methods for computing invariants \\ \qquad \qquad (e.g. Rintanen, AAAI'00 or Alcázar \&  Torralba, ICAPS'15).\\
\pause
The ones that interest us here are binary disjunctive invariants:
\[\ell_1 \vee \ell_2\]
\pause
	What to do with an invariant $\ell_1 \vee \ell_2$?\pause
	Add it to every timestep $t$ as $\ell_1^t \vee \ell_2^t$.\\[\baselineskip]
\pause These clauses are probably helpful for SAT-based planner, Rintanen, KR'08
\end{frame}


\usetikzlibrary{arrows,chains,decorations.pathmorphing,decorations.shapes,shapes.geometric,positioning,calc,shapes,scopes,backgrounds,fit,matrix,trees}
\usetikzlibrary{decorations.pathreplacing}
\tikzset{ onslide/.code args={<#1>#2}{\only<#1>{\pgfkeysalso{#2}}} }
% \definecolor{uc_prime}{RGB}{0,153,0}  % Standard-Farbe von Tasks
% \definecolor{uc_second}{RGB}{1,1,1}   % insertion arrows
% \definecolor{uc_accent}{RGB}{20,80,20}   % Lösungen

\newcommand{\LabelForInit} {init}
\newcommand{\LabelForGoal} {goals}

% \definecolor{uc_prime}{RGB}{125,154,170}  % modifiziert
% \definecolor{uc_second}{RGB}{163,38,56}   % insertion arrows
% \definecolor{uc_accent}{RGB}{169,162,141} % Lösungen

\pgfarrowsdeclare{biggertip}{biggertip}{  
  \setlength{\arrowsize}{0.03em}  
  \addtolength{\arrowsize}{.5\pgflinewidth}  
  \pgfarrowsrightextend{0}  
  \pgfarrowsleftextend{-5\arrowsize}  
}{  
  \setlength{\arrowsize}{0.03em}  
  \addtolength{\arrowsize}{.5\pgflinewidth}  
  \pgfpathmoveto{\pgfpoint{-5\arrowsize}{4\arrowsize}}  
  \pgfpathlineto{\pgfpointorigin}  
  \pgfpathlineto{\pgfpoint{-5\arrowsize}{-4\arrowsize}}  
  \pgfusepathqstroke  
}

\tikzset{
  math/.style={execute at begin node=$,execute at end node=$},
  tn/.style={draw,rounded corners=0.8em,minimum width=8.5em,fill=white},      %task network
  solution/.style={fill=uc_accent!30},
  modified/.style={fill=uc_prime!30},
  cut/.style={draw=black,text=black,fill=uc_prime!30},
  uncut/.style={draw=black,text=black,fill=white},
  task/.style={thick,math,minimum width=2em,draw,fill=white,text depth=0em,text height=0.7em},
  primitive/.style={rounded corners=0.2em},
  abstract/.style={rounded corners=0em},
  abstractm/.style={abstract,modified},
  primm/.style={primitive,modified},
  tprim/.style={task,primitive},                                        % primitive task
  tabstr/.style={task,abstract},                                        % abstract task
  insertion/.style={-latex,dashed,draw=uc_second,very thick},           % insertion arrows
  decomposition/.style={-biggertip,very thick,draw},
  oc/.style={-latex,thick},                                             % ordering constraints
  method/.style={rounded corners,draw=\ColorTop,ultra thick,rectangle},
  onslide/.code args={<#1>#2}{\only<#1>{\pgfkeysalso{#2}}}
}

\def\emptyDefault#1#2{\ifx&#1&%
#2\else#1\fi}

\newcommand{\TaskSeq}[1]{
   \path[start chain=1,every on chain/.style={join=by oc},node distance=0.9em]
     \foreach \x/\attr in {#1} {
       node[task,
          rectangle,
          on chain=1,
          primitive,
          \attr] {\x}};
}

\newcommand{\MInsert}[1] {Insertion: $#1$}
\newcommand{\MDec}[1]    {Decomposition: $#1$}

\newcommand{\StackTN}[2] {
  \tikz[start chain=going below, node distance=0]{
    \node[on chain,draw=none] {\tikz{\TaskSeq{#1}}};
    \node[on chain,draw=none] {\tikz{\TaskSeq{#2}}};
  }
}

\newcommand{\GF}{}    %comes after a symbol from the first grammar
\newcommand{\GS}{'}    %comes after a symbol from the second grammar

\newcommand{\BlueBullets}{
  \renewcommand{\labelitemi}{\textcolor{uc_prime}{$\bullet$}}
  \renewcommand{\labelitemii}{\textcolor{uc_prime}{$\blacktriangleright$}}
}
\newcommand{\BrownBullets}{\renewcommand{\labelitemi}{\textcolor{uc_accent}{$\bullet$}}}


\tikzstyle{heading}=[rounded corners=0.3em,inner xsep=0.6em,right=10pt,fill=uc_prime,text=white, text depth=0.1em,font=\bfseries]
\tikzstyle{box}=[
  rounded corners=1em,
  inner sep=1em,
  inner ysep=1.5em,
  fill=uc_prime!20,
  line width=5pt,
  draw=uc_prime!40,
  execute at begin node=\BlueBullets
]
\tikzstyle{comment}=[
  box,
  fill=uc_accent!20,
  draw=uc_accent!40,
  execute at begin node=\BrownBullets
]



\newcommand{\LegItem}[3]     {\node {$#2 \rightarrow$}; \pgfmatrixnextcell \node[tn] {\tikz{#3}};}
\newcommand{\LegListItem}[3] {\LegItem{#1}{#2}{\TaskSeq{#3}}}


%%%%%%%%%%%%%%%%%%%%%%%%%%%%%%%%%%%%%%%%%%%%%% symbols

\def\foldedpaper#1{
    \tikz[scale=#1,line width={#1*1pt}]{
        \def\a{1.41} % relative height
        \def\b{0.2}  % relative height/width of corner
        \def\c{0.1}  % relative margin width (on either side)
        \def\d{0.05} % vertical offset of lines
        \def\N{6}    % number of lines
        \draw         (0,0)
                --  ++(-1,0)
                --  ++(0,\a)
                --  ++(1-\b,0)
                --  ++(\b,-\b)
                -- cycle;
        \foreach \lnum in {1,...,\N}{
            \pgfmathsetmacro\yline{\a-\d-\lnum*\a/(\N+1)}
            \draw (-1+\c,\yline) -- (-\c,\yline);
        }
        \draw[fill=white] (0,\a-\b) -- ++(-\b,0) -- ++ (0,\b);
    }
}

% #1 number of teeths
% #2 radius intern
% #3 radius extern
% #4 angle from start to end of the first arc
% #5 angle to decale the second arc from the first
% #6 inner radius to cut off

\newcommand{\gear}[7]{
\tikz[scale=#7,line width={#7*1pt}]{%
	\fill[even odd rule]
  (0:#2)
  \foreach \i [evaluate=\i as \n using {\i-1)*360/#1}] in {1,...,#1}{%
    arc (\n:\n+#4:#2) {[rounded corners=1.5pt] -- (\n+#4+#5:#3)
    arc (\n+#4+#5:\n+360/#1-#5:#3)} --  (\n+360/#1:#2)
  }%
  (0,0) circle[radius=#6]
  }
}


\newcommand{\manfig}[1]{
\tikz[scale=#1,line width={#1*1pt}]{%
\node[circle,fill,minimum size=5mm] (head) {};
\node[rounded corners=2pt,minimum height=1.3cm,minimum width=0.4cm,fill,below = 1pt of head] (body) {};
\draw[line width=1mm,round cap-round cap] ([shift={(2pt,-1pt)}]body.north east) --++(-90:6mm);
\draw[line width=1mm,round cap-round cap] ([shift={(-2pt,-1pt)}]body.north west)--++(-90:6mm);
\draw[thick,white,-round cap] (body.south) --++(90:5.5mm);
}
}




\section{Sequential Classical Planning in SAT}


\begin{frame}<1>{Classical Planning via SAT [Kautz\&Selman'92]}
  \vspace{-1cm}
	\begin{center}
      \begin{tikzpicture}
  \coordinate (nodeDifferenceH) at (1,0);
  \coordinate (nodeDifferenceV) at (0,.35);
  
  \node at (21,-12) {};
  \node at (10,-7.5) {};
  
  \node<1->[S] (S1) at (11.5,-9.5) {};
  \node<1->[S] (S2) at ($(S1)+1*(nodeDifferenceH)$) {};
  \node<1->[S] (S3) at ($(S1)+2*(nodeDifferenceH)$) {};
  \node<1->[S] (S4) at ($(S1)+3*(nodeDifferenceH)$) {};
  \node<1->[S] (S5) at ($(S1)+4*(nodeDifferenceH)$) {};
  \node<1->[S] (S6) at ($(S1)+5*(nodeDifferenceH)$) {};
  \node<1->[S] (S7) at ($(S1)+6*(nodeDifferenceH)$) {};
  \node<1->[S] (S8) at ($(S1)+7*(nodeDifferenceH)$) {};
  \draw<1->[fill] (10.5,-10.) rectangle (10.45,-9.);
  \node<1-> at (10.45,-8.8) {\tiny $I$};
  \draw<1->[fill] (19.5,-10.) rectangle (19.45,-9.);
  \node<1-> at (19.45,-8.8) {\tiny $G$};
  
  \draw<1->[->] (10.5,-9.5) -- (S1);
  \draw<1->[->] (S1) edge [] (S2);
  \draw<1->[->] (S2) edge [] (S3);
  \draw<1->[->] (S3) edge [] (S4);
  \draw<1->[->] (S4) edge [] (S5);
  \draw<1->[->] (S5) edge [] (S6);
  \draw<1->[->] (S6) edge [] (S7);
  \draw<1->[->] (S7) edge [] (S8);
  \draw<1->[->,shorten >=.5mm] (S8) edge [] (19.5,-9.5);
  
  % new: 1-3 was: 2-
  \node<1-3> at ($(11,-10.5)+1/2*(nodeDifferenceH)$) (label-S0) {\tiny $s_{1}$};
  \foreach \x in {1,...,8} {
    \edef\xPlusOne{\x}
    \pgfmathparse{int(\xPlusOne+1)}
    \edef\xPlusOne{\pgfmathresult}
    \node<1-3> at ($(label-S0)+\x*(nodeDifferenceH)$)   {\tiny $s_{\xPlusOne}$};
  }
  \node<1-3> at ($(label-S0)-(nodeDifferenceH)$) {\tiny $s_{I}=s_0$};
  
  % new: 1-2 was: 3-
  \node<1-2> at ($(11.5,-9.25)-1/2*(nodeDifferenceH)$) (label-A1) {\tiny $a_{1}$};
  \foreach \x in {1,...,8} {
    \edef\xPlusOne{\x}
    \pgfmathparse{int(\xPlusOne+1)}
    \edef\xPlusOne{\pgfmathresult}
    \node<1-2> at ($(label-A1)+\x*(nodeDifferenceH)$)   {\tiny $a_{\xPlusOne}$};
  }
  
  \foreach \x in {11.5,...,19.5} {
    \edef\upperIndex{\x}
    \pgfmathparse{int(\upperIndex-10.5)}
    \edef\upperIndex{\pgfmathresult}
    \node<3-> at ($(\x,-9.25)-1/2*(nodeDifferenceH)$) (label-A\upperIndex) {\tiny $a_{1}^{\upperIndex}$};
    \foreach \y in {1,...,4} {
      \edef\yPlusOne{\y}
      \pgfmathparse{int(\yPlusOne+1)}
      \edef\yPlusOne{\pgfmathresult}
      \node<3-> at ($(label-A\upperIndex)+\y*(nodeDifferenceV)$)    {\tiny $a_{\yPlusOne}^{\upperIndex}$};
    }
  }
  
  \foreach \x in {10,...,19} {
    \edef\upperIndex{\x}
    \pgfmathparse{int(\upperIndex-10)}
    \edef\upperIndex{\pgfmathresult}
    \node<4-> at ($(\x,-10.5)+1/2*(nodeDifferenceH)$) (label-V\upperIndex) {\tiny $v_{1}^{\upperIndex}$};
     \foreach \y in {1,...,3} {
      \edef\yPlusOne{\y}
      \pgfmathparse{int(\yPlusOne+1)}
      \edef\yPlusOne{\pgfmathresult}
      \node<4-> at ($(label-V\upperIndex)-\y*(nodeDifferenceV)$)    {\tiny $v_{\yPlusOne}^{\upperIndex}$};
    }
  }
\end{tikzpicture}

	\end{center}
	A plan is just a sequence of state transitions.
	\begin{itemize}
		\item ``Mechanics'' is identical in all timesteps.
		\item Just model one timestep and copy'n'paste.
		\item Edge constraints!
	\end{itemize}
\end{frame}




\begin{frame}{Decision Variables}
	\begin{center}
      \begin{tikzpicture}
  \coordinate (nodeDifferenceH) at (1,0);
  \coordinate (nodeDifferenceV) at (0,.35);
  
  \node at (21,-12) {};
  \node at (10,-7.5) {};
  
  \node<1->[S] (S1) at (11.5,-9.5) {};
  \node<1->[S] (S2) at ($(S1)+1*(nodeDifferenceH)$) {};
  \node<1->[S] (S3) at ($(S1)+2*(nodeDifferenceH)$) {};
  \node<1->[S] (S4) at ($(S1)+3*(nodeDifferenceH)$) {};
  \node<1->[S] (S5) at ($(S1)+4*(nodeDifferenceH)$) {};
  \node<1->[S] (S6) at ($(S1)+5*(nodeDifferenceH)$) {};
  \node<1->[S] (S7) at ($(S1)+6*(nodeDifferenceH)$) {};
  \node<1->[S] (S8) at ($(S1)+7*(nodeDifferenceH)$) {};
  \draw<1->[fill] (10.5,-10.) rectangle (10.45,-9.);
  \node<1-> at (10.45,-8.8) {\tiny $I$};
  \draw<1->[fill] (19.5,-10.) rectangle (19.45,-9.);
  \node<1-> at (19.45,-8.8) {\tiny $G$};
  
  \draw<1->[->] (10.5,-9.5) -- (S1);
  \draw<1->[->] (S1) edge [] (S2);
  \draw<1->[->] (S2) edge [] (S3);
  \draw<1->[->] (S3) edge [] (S4);
  \draw<1->[->] (S4) edge [] (S5);
  \draw<1->[->] (S5) edge [] (S6);
  \draw<1->[->] (S6) edge [] (S7);
  \draw<1->[->] (S7) edge [] (S8);
  \draw<1->[->,shorten >=.5mm] (S8) edge [] (19.5,-9.5);
  
  % new: 1-3 was: 2-
  \node<1-3> at ($(11,-10.5)+1/2*(nodeDifferenceH)$) (label-S0) {\tiny $s_{1}$};
  \foreach \x in {1,...,8} {
    \edef\xPlusOne{\x}
    \pgfmathparse{int(\xPlusOne+1)}
    \edef\xPlusOne{\pgfmathresult}
    \node<1-3> at ($(label-S0)+\x*(nodeDifferenceH)$)   {\tiny $s_{\xPlusOne}$};
  }
  \node<1-3> at ($(label-S0)-(nodeDifferenceH)$) {\tiny $s_{I}=s_0$};
  
  % new: 1-2 was: 3-
  \node<1-2> at ($(11.5,-9.25)-1/2*(nodeDifferenceH)$) (label-A1) {\tiny $a_{1}$};
  \foreach \x in {1,...,8} {
    \edef\xPlusOne{\x}
    \pgfmathparse{int(\xPlusOne+1)}
    \edef\xPlusOne{\pgfmathresult}
    \node<1-2> at ($(label-A1)+\x*(nodeDifferenceH)$)   {\tiny $a_{\xPlusOne}$};
  }
  
  \foreach \x in {11.5,...,19.5} {
    \edef\upperIndex{\x}
    \pgfmathparse{int(\upperIndex-10.5)}
    \edef\upperIndex{\pgfmathresult}
    \node<3-> at ($(\x,-9.25)-1/2*(nodeDifferenceH)$) (label-A\upperIndex) {\tiny $a_{1}^{\upperIndex}$};
    \foreach \y in {1,...,4} {
      \edef\yPlusOne{\y}
      \pgfmathparse{int(\yPlusOne+1)}
      \edef\yPlusOne{\pgfmathresult}
      \node<3-> at ($(label-A\upperIndex)+\y*(nodeDifferenceV)$)    {\tiny $a_{\yPlusOne}^{\upperIndex}$};
    }
  }
  
  \foreach \x in {10,...,19} {
    \edef\upperIndex{\x}
    \pgfmathparse{int(\upperIndex-10)}
    \edef\upperIndex{\pgfmathresult}
    \node<4-> at ($(\x,-10.5)+1/2*(nodeDifferenceH)$) (label-V\upperIndex) {\tiny $v_{1}^{\upperIndex}$};
     \foreach \y in {1,...,3} {
      \edef\yPlusOne{\y}
      \pgfmathparse{int(\yPlusOne+1)}
      \edef\yPlusOne{\pgfmathresult}
      \node<4-> at ($(label-V\upperIndex)-\y*(nodeDifferenceV)$)    {\tiny $v_{\yPlusOne}^{\upperIndex}$};
    }
  }
\end{tikzpicture}

	\end{center}
	\visible<2->{We only need two types of decision variables!}
	\begin{enumerate}
		\item<3-> $a_{i}^t$ -- Action $i$ is executed at time $t$.
		\item<4-> $v_{i}^t$ -- State variable $i$ is true at time $t$.
	\end{enumerate}
\end{frame}




\begin{frame}<2->{Overall Formula}
	\vspace{-0.1cm}
	\begin{center}
      \begin{tikzpicture}
  \coordinate (nodeDifferenceH) at (1,0);
  \coordinate (nodeDifferenceV) at (0,.35);
  
  \node at (21,-12) {};
  \node at (10,-7.5) {};
  
  \node<1->[S] (S1) at (11.5,-9.5) {};
  \node<1->[S] (S2) at ($(S1)+1*(nodeDifferenceH)$) {};
  \node<1->[S] (S3) at ($(S1)+2*(nodeDifferenceH)$) {};
  \node<1->[S] (S4) at ($(S1)+3*(nodeDifferenceH)$) {};
  \node<1->[S] (S5) at ($(S1)+4*(nodeDifferenceH)$) {};
  \node<1->[S] (S6) at ($(S1)+5*(nodeDifferenceH)$) {};
  \node<1->[S] (S7) at ($(S1)+6*(nodeDifferenceH)$) {};
  \node<1->[S] (S8) at ($(S1)+7*(nodeDifferenceH)$) {};
  \draw<1->[fill] (10.5,-10.) rectangle (10.45,-9.);
  \node<1-> at (10.45,-8.8) {\tiny $I$};
  \draw<1->[fill] (19.5,-10.) rectangle (19.45,-9.);
  \node<1-> at (19.45,-8.8) {\tiny $G$};
  
  \draw<1->[->] (10.5,-9.5) -- (S1);
  \draw<1->[->] (S1) edge [] (S2);
  \draw<1->[->] (S2) edge [] (S3);
  \draw<1->[->] (S3) edge [] (S4);
  \draw<1->[->] (S4) edge [] (S5);
  \draw<1->[->] (S5) edge [] (S6);
  \draw<1->[->] (S6) edge [] (S7);
  \draw<1->[->] (S7) edge [] (S8);
  \draw<1->[->,shorten >=.5mm] (S8) edge [] (19.5,-9.5);
  
  % new: 1-3 was: 2-
  \node<1-3> at ($(11,-10.5)+1/2*(nodeDifferenceH)$) (label-S0) {\tiny $s_{1}$};
  \foreach \x in {1,...,8} {
    \edef\xPlusOne{\x}
    \pgfmathparse{int(\xPlusOne+1)}
    \edef\xPlusOne{\pgfmathresult}
    \node<1-3> at ($(label-S0)+\x*(nodeDifferenceH)$)   {\tiny $s_{\xPlusOne}$};
  }
  \node<1-3> at ($(label-S0)-(nodeDifferenceH)$) {\tiny $s_{I}=s_0$};
  
  % new: 1-2 was: 3-
  \node<1-2> at ($(11.5,-9.25)-1/2*(nodeDifferenceH)$) (label-A1) {\tiny $a_{1}$};
  \foreach \x in {1,...,8} {
    \edef\xPlusOne{\x}
    \pgfmathparse{int(\xPlusOne+1)}
    \edef\xPlusOne{\pgfmathresult}
    \node<1-2> at ($(label-A1)+\x*(nodeDifferenceH)$)   {\tiny $a_{\xPlusOne}$};
  }
  
  \foreach \x in {11.5,...,19.5} {
    \edef\upperIndex{\x}
    \pgfmathparse{int(\upperIndex-10.5)}
    \edef\upperIndex{\pgfmathresult}
    \node<3-> at ($(\x,-9.25)-1/2*(nodeDifferenceH)$) (label-A\upperIndex) {\tiny $a_{1}^{\upperIndex}$};
    \foreach \y in {1,...,4} {
      \edef\yPlusOne{\y}
      \pgfmathparse{int(\yPlusOne+1)}
      \edef\yPlusOne{\pgfmathresult}
      \node<3-> at ($(label-A\upperIndex)+\y*(nodeDifferenceV)$)    {\tiny $a_{\yPlusOne}^{\upperIndex}$};
    }
  }
  
  \foreach \x in {10,...,19} {
    \edef\upperIndex{\x}
    \pgfmathparse{int(\upperIndex-10)}
    \edef\upperIndex{\pgfmathresult}
    \node<4-> at ($(\x,-10.5)+1/2*(nodeDifferenceH)$) (label-V\upperIndex) {\tiny $v_{1}^{\upperIndex}$};
     \foreach \y in {1,...,3} {
      \edef\yPlusOne{\y}
      \pgfmathparse{int(\yPlusOne+1)}
      \edef\yPlusOne{\pgfmathresult}
      \node<4-> at ($(label-V\upperIndex)-\y*(nodeDifferenceV)$)    {\tiny $v_{\yPlusOne}^{\upperIndex}$};
    }
  }
\end{tikzpicture}

	\end{center}
	\vspace{-.5cm}
	\visible<3->{Constraints to check:}
	\begin{itemize}
		\item<3-> Correctly applying actions at each time step ($\tau$).
		\item<4-> $I$ and $G$ must be respected.
	\end{itemize}
	\vspace{-.1cm}
	\begin{align*}
		\visible<3->{\mathcal F =} \visible<3->{\bigwedge_{t=0}^{k-1} \tau(t)} \visible<4->{\wedge}
		\onslide<4->{\bigwedge_{v_i \in I} v_i^0 \wedge \bigwedge_{v_i \in V \setminus I} \hspace{-0.2cm}\neg v_i^0 \wedge \bigwedge_{v_i \in G} v_i^{k}}\visible<3->{\quad \text{here: $k=9$}}
	\end{align*}
\end{frame}



\begin{frame}{Classical Planning via SAT}
  \begin{center}
  \begin{tikzpicture}
    \node[S] (S1) at (-2,0) {};
    \node[above of=S1,yshift=-.5cm] {$s$};
    \node[S] (S2) at (0,0)  {};
    \node[above of=S2,yshift=-.5cm] {$s'$};
    \draw[->] (S1) --  node[midway,above] {$a$} (S2);
  \end{tikzpicture}
  \end{center}
  Constraints to check by $\tau(t)$:
  \begin{itemize}[<+(1)->]
	\item[$F_1$] Preconditions must hold (in $s$).
	\item[$F_2$] Effects must occur (in $s'$).
    \item[$F_3$] Unaffected state variables stay unchanged.
    \item[$F_4$] At most one action per timestep.
    \item[$F_5$] At least one action per timestep. \visible<+->{Necessary?} \visible<+->{\textbf{No.}}
  \end{itemize}
\end{frame}






\begin{frame}{Classical Planning via SAT}
 	\begin{itemize}
		\item Preconditions must hold:
\visible<2->{\[F_1 = \bigwedge_{a \in A} a^{t+1} \rightarrow \bigwedge_{v \in pre(a)} v^t\]}
		\item Effects must occur:
			\begin{align*}\visible<3->{
					F_2 = &\left[\bigwedge_{a \in A} a^{t+1} \rightarrow	\bigwedge_{v \in add(a)} v^{t+1}\right]  \hspace{0.75cm} \wedge \\
					&\left[\bigwedge_{a \in A} a^{t+1} \rightarrow \bigwedge_{v \in del(a)} \neg v^{t+1}\right]}
			\end{align*}
	\end{itemize}
\end{frame}






\begin{frame}{Classical Planning via SAT}
 	\begin{itemize}
	\item Variables not affected by the executed action must stay the same.
		\begin{itemize}
			\item<2->[$\rightarrow$] Frame Problem!
		\end{itemize}
			\visible<3->{\begin{align*}F_3 = &\bigwedge_{v \in V} \left[(\neg v^{t} \wedge v^{t+1}) \rightarrow \hspace{-0.2cm}\bigvee_{a \in A \text{ with } v \in add(a)} \hspace{-0.8cm} a^{t+1} \hspace{0.4cm}\right] \wedge \\
				&\bigwedge_{v \in V}  \left[(v^{t} \wedge \neg v^{t+1}) \rightarrow \hspace{-0.2cm}\bigvee_{a \in A \text{ with } v \in del(a)} \hspace{-0.8cm} a^{t+1} \hspace{0.4cm}\right]
		\end{align*}}
		\item Only one action at a time\footnote{Various encodings exist. Typical are pairwise and bitwise. See \url{https://pysathq.github.io/docs/html/api/card.html} for a handy reference}:
			\visible<4->{\[F_4 = \sum_{a \in A} a^t \leq 1\]}
	\end{itemize}
\end{frame}


\begin{frame}{Extended Expressiveness}
 	\begin{itemize}
		\item Conditional Effects
 		\begin{itemize}
			\item at least since Madagascar, Rintanen, Heljanko, Niemelä, JAIR'06
		\end{itemize}
		\vspace{0.2cm}
		\pause
		\item Axioms / Derived Predicates
 		\begin{itemize}
			\item AxSAT, Behnke, Speck, Gnad, JELIA'25
		\end{itemize}
	\end{itemize}
\end{frame}


\begin{frame}{Other Base Encodings}
 	\begin{itemize}
		\item Partial-Order Causal-Link (POCL) encodings
 		\begin{itemize}
			\item Kautz, McAllester, Selman, KR'96\\ Depending on implementation cubic or quadratic in plan length
		\end{itemize}
		\vspace{0.2cm}
		\pause
		\item Encoding for SAS+ tasks:
 		\begin{itemize}
			\item SASE, Huang, Chen, Zhang, JAIR'12\\ quadratic (vars) and cubic (clauses) in $|D(v)|$
		\end{itemize}
		\vspace{0.2cm}
		\pause
		\item Encodings of Lifted Planning:
 		\begin{itemize}
			\item LiSAT, based on POCL, Höller\&Behnke, ICAPS'22
				\\ quadratic in plan length
			\item based on QBF, Shaik \& van de Pol, ICAPS'22
				\\ quadratic in plan length
		\end{itemize}
	\end{itemize}
\end{frame}







%
%
%
%\subsection{At-most-one}
%
%
%\newcommand{\TwoRowTab}[2]{\begin{tabular}{c}\footnotesize \ensuremath{#1}\\[-.25em]\footnotesize \ensuremath{#2}\end{tabular}}
%\begin{frame}{At-most-one}
%	Given a set of decision variables $X = \{x_1, \dots, x_{|X|}\}$. Find a set of clauses that, if satisfied, will ensure that at most one $x \in X$ is true.\\[\baselineskip]
%	\visible<2->{Naive encoding:}
%	\visible<2->{\[
%			\bigwedge_{x_1 \in X} \bigwedge_{x_2 \in X \setminus \{x_1\}} \underbrace{\neg x_1 \vee \neg x_2}_{\TwoRowTab{(x_1\Rightarrow \neg x_2)\ \wedge}{(x_2\Rightarrow \neg x_1)}}
%	\]}
%\end{frame}
%
%
%
%
%
%
%
%\begin{frame}{At-most-one}
%	Idea: Introduce new variables!
%	\pause
%	\begin{align*}
%		f_i &\text{ -- from index $i$ on all $x_i$ will be false}\\
%		&\text{i.e. it is \textbf{f}orbidden to use any $x_i$ after $i$}
%	\end{align*}
%	\pause
%	 Sequential encoding:
%	\begin{align*}
%		&\bigwedge_{i=1}^{|X|-1} \underbrace{\neg x_i \vee f_i}_{x_i \Rightarrow f_i} 
%		&\bigwedge_{i=2}^{|X|-1} \underbrace{\neg f_{i-1} \vee f_{i}}_{f_{i-1} \Rightarrow f_{i}}
%	\end{align*}
%	\begin{align*}
%		&\bigwedge_{i=1}^{|X|} \underbrace{\neg x_i \vee \neg f_{i-1}}_{\TwoRowTab{(x_i \Rightarrow \neg f_{i-1})\ \wedge}{(f_{i-1} \Rightarrow \neg x_i)}}
%	\end{align*}
%	%\pause
%\end{frame}
%
%
%
%
%
%
%\begin{frame}{At-most-one}
%	Maybe this is a bit much ...
%	\pause
%	\[n_i \text{ -- bit $i$ (0-index) of a $\lceil\log(|X|)\rceil$-digit binary number if one}\]
%	\pause
%	Binary encoding:
%	\begin{align*}
%		\neg x_i \vee n_j \quad \quad \text{ if } \frac{i}{2^{j}} \mod 2 = 1\\
%		\neg x_i \vee \neg n_j \quad \quad \text{ if } \frac{i}{2^{j}} \mod 2 = 0\\
%	\end{align*}
%\end{frame}
%
%
%
%
%%\renewcommand{\citeA}{1}
%
%\begin{frame}{Different AMO Implementations}
%	Overview\footnote{Frisch and Giannaros; SAT Encodings of the At-Most-k Constraint -- Some Old, Some New, Some Fast, Some Slow; 2010}
%	
%	\begin{center}
%	\begin{tabular}{l|l|l|l|l}
%		encoding & \#clauses & \#new variables \\ \hline \hline
%		binomial & $n^{2^{\phantom{1}}}$ & 0 \\\hline
%		binary & $n \log n$ & $\log n$\\\hline
%		sequential & $3n$ & $n$\\\hline
%		commander & $\frac{7}{2}n^{\phantom{1^{\phantom{1}}}}$ & $\frac{n}{2}$\\\hline
%		product & $2(n + n^{\frac{^{\phantom{1}}1^{\phantom{1}}}{m+1}})$ & $2n^{\frac{1}{2}}$\\
%	\end{tabular}\ \\[1em]
%	where $n$ is the number of atoms, i.e., $|X|$
%    \end{center}
%
%
%\end{frame}
%
%
%%\againframe{BoundIteration}
%
%
%\begin{frame}{Classical Planning via SAT}
%	\begin{center}
%		There are \textbf{a lot} of improvements to this formula.
%	\end{center}
%	\begin{itemize}[<+(1)->]
%		\item Invariants.
%		\item $\forall$-step semantics.
%		\item $\exists$-step semantics.
%	\end{itemize}
%\end{frame}
